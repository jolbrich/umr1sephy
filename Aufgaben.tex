\documentclass[a4paper]{scrartcl}

\usepackage[ngerman]{babel}
\usepackage[utf8]{inputenc}

\usepackage{hyperref}

\usepackage{amsmath}

\usepackage{units}

\usepackage{enumitem}

\usepackage{longtable}

\usepackage{todonotes}

\renewcommand{\labelenumi}{\alph{enumi})}

\begin{document}

\section{Thematisch sortierte Aufgaben}

\begin{longtable}{ll}
  Mechanik & Auftrieb/Archimedisches Prinzip \\
  & Erzwungene Präzession/Kollergang\\
  & Myonen\\
  & Rakentenstart\\
  & Relativitätstheorie\\
  & Reise nach Australien\\
  & Rotierende Raumstation\\
  & Segelboot\\
  & Strahlungsdruck \\
  & Thermische Bewegung von Neutronen\\
  & Walze auf schiefer Ebene\\
  & Arbeit und Leistung am Wassertank\\
  & Wurf in beschleunigtem Zug\\
  & Zeppelin\\
  Elektrodynamik & Belasteter Spannungsteiler\\
  & Spule mit Eisenkern\\
  & Elektronen im E- und B-Feld\\
  & e/m-Bestimmung\\
  & Energie im Kondensator\\
  & Geschwindigkeitsfilter für Protonen\\
  & Kondensator mit Dielektrika\\
  & Intensität einer Lampe\\
  & Kirchhoff\\
  Thermodynamik & Eis im Kühlschrank \\
  & Schwarzer Strahler\\
  & (*) Thermische Bewegung von Neutronen\\
  & Wärmekapazität\\
  & Wärmeleitung\\
  & Wärmekraftmaschine\\
  & (*) Zeppelin\\
  Optik + Quanten & Gitterspektralapparat \\
  & Lichtleitfaser \\
  & Michelson-Interferometer \\
  & Newtonsche Ringe \\
  & Punktförmige Lichtquelle \\
  & Reflexion an dünner Schicht \\
  & (*) Strahlungsdruck \\
  & Totalreflexion am Wassertank\\
  & Strahlengänge\\
  Atomphysik & (*) Elektronen im E- und B-Feld\\
  & Gravitationsatom \\
  & Isotopie \\
  & Lamorfrequenz \\
  & Millikan-Versuch \\
  & Wasserstoffspektrum \\
  & Zyklotron\\
  Kernphysik & (*) Isotopie \\
  & (*) Myonen\\
  & Neutronen in Wasser \\
  & (*) Thermische Bewegung von Neutronen\\
  Wellen & Dopplereffekt \\
  & Harmonischer Oszillator \\
  & Schwinger in Wasser/Gas\\
  & Stehende Welle\\
\end{longtable}

\section{Aufgaben}

\subsection{Auftrieb/Archimedisches Prinzip}
\label{aufg:Auftrieb}
In ein Wasserbecken (Grundfläche $A = \unit[5]{m} \cdot \unit[5]{m}$, Füllhöhe $h_0 = \unit[1.5]{m}$) wird ein Boot gelassen ($m_\text{B} = \unit[40]{kg}$), in dem ein Stein liegt ($m_\text{St} = \unit[10]{kg}$, $\varrho_\text{St} = \unit[3]{\frac{kg}{dm^3}})$.

\begin{enumerate}[noitemsep]
  \item Wie ändert sich die Höhe des Wasserspiegels?
  \item Wie ändert sich die Höhe des Wasserspiegels, wenn der Stein ins Wasser geworfen wird?
\end{enumerate}

\hyperref[lsg:Auftrieb]{zur Lösung}

\subsection{Erzwungene Präzession/Kollergang}
\label{aufg:Kollergang}
Ein Rad ($r = \unit[0.5]{m}$, $m=\unit[1500]{kg}$) ist an einer $R = \unit[2]{m}$ langen Achse so gelagert, dass es beim Abrollen eine Kreisbahn beschreibt. Die Frequenz dieses Umlaufes beträgt $f_\text{P} = \unit[0.5]{Hz}$. Mit welcher Kraft drückt die Scheibe auf die Unterlage? Vektorielle Skizze!

\hyperref[lsg:Kollergang]{zur Lösung}

\subsection{Myonen}
\label{aufg:Myonen}
Durch Höhenstrahlung entstehen in $\unit[10]{km}$ Höhe über der Erdoberfläche Myonen (Elementarteilchen, negativ geladen, etwa 200mal schwerer als ein Elektron). Sie bewegen sich mit der Geschwindigkeit $v = 0.9995c$ und zerfallen mit einer mittleren Lebensdauer $T_{1/2} = \unit[2\cdot 10^{-6}]{s}$.
\begin{enumerate}[noitemsep]
  \item Wie groß ist die mittlere Lebensdauer im Bezugssystem Erde, welche Reichweite ergibt sich daraus?
  \item Wie groß ist die Reichweite im Bezugssystem des Myons?
  \item Wie lang ist die Strecke zwischen Entstehungsort und Erdoberfläche im Bezugssystem des Myons?
  \item Erreichen die Myonen die Erde?
\end{enumerate}

\hyperref[lsg:Myonen]{zur Lösung}

\subsection{Raketenstart}
\label{aufg:Raketenstart}
Eine Rakete mit Startmasse $m_0 =\unit[100]{t}$ startet von der Erdoberfläche senkrecht nach oben. Die Verbrennungsgase treten mit $v = \unit[4000]{\frac{m}{s}}$ relativ zur Rakete aus.

Die Erdbeschleunigung kann über die gesamte Höhe als konstant angesehen werden, Luftreibung vernachlässigen.

\begin{enumerate}[noitemsep]
  \item Welcher Gasausstoß $n_1 = -\frac{\text{d}m}{\text{d}t}$ ist nötig, damit die Rakete gerade eben schweben kann?
  \item Nach welcher Zeit $t_\text{e}$ hat die Rakete noch die halbe Startmasse, wenn der Gasausstoß $n_2 = \unit[500]{\frac{kg}{s}}$ beträgt?
  \item Wie groß ist die Beschleunigung zum Zeitpunkt $t_\text{e}$?
  \item Welche Geschwindigkeit hat die Rakete zum Zeitpunkt $t_\text{e}$?
  \item Wie hoch ist die Rakete gestiegen?
  \item Welche Masse hat die Rakete, wenn sie die erste kosmische Geschwindigkeit $v_1 = \unit[7.91]{\frac{km}{s}}$, die zweite kosmische Geschwindigkeit $v_2 = \unit[11.18]{\frac{km}{s}}$? Wie lange dauert es? (Hinweis: hier muss die Erdanziehungskraft ignoriert werden.)
\end{enumerate}

\hyperref[lsg:Raketenstart]{zur Lösung}

\subsection{Relativitätstheorie}
\label{aufg:Relativitaetstheorie}
Auf einen Körper der Masse $m=\unit[1]{g}$ wirke eine konstante Kraft von $\unit[10]{N}$.
\begin{enumerate}[noitemsep]
  \item Wie groß ist die Beschleunigung?
  \item Wie groß ist die Beschleunigung, wenn der Körper drei Viertel der Lichtgeschwindigkeit erreicht hat?
\end{enumerate}

\hyperref[lsg:Relativitaetstheorie]{zur Lösung}

\subsection{Reise nach Australien (Weg durch Erdmittelpunkt)}
\label{aufg:Australien}
Man betrachte einen durch die Erde gebohrten Tunnel und ein sich darin Reibungsfrei bewegendes Objekt. Gesucht sind
\begin{enumerate}[noitemsep]
  \item "`Reisedauer"' für eine Durchquerung
  \item Erreichte Höchstgeschwindigkeit
  \item Zeit und Ort, wenn man seinen zu Beginn des Falls ausgestoßenen Angstschrei einholt.
\end{enumerate}
Benötigte Konstanten: Gravitationskonstante $\gamma = \unit[6.674 \cdot 10^{-11}]{\frac{m^3}{kg \cdot s^2}}$, Erdradius $R \approx \unit[6365]{km}$, mittlere Dichte der Erde $\varrho = \unit[5515]{\frac{kg}{m^3}}$, Schallgeschwindigkeit $v_\text{S} = \unit[340]{\frac{m}{s}}$.

\hyperref[lsg:Australien]{zur Lösung}

\subsection{Rotierende Raumstation}
\label{aufg:Raumstation}
Eine zylinderförmige Raumstation mit dem Trägheitsmoment $J = \unit[0.4 \cdot 10^9]{kg\cdot m^2}$ entlang der Zylinderachse rotiert um diese mit einer Umlaufzeit von $T_1 = \unit[1]{min}$. Der Radius beträgt $\unit[50]{m}$. Dort befinden sich vier tangential angebrachte Triebwerke.
\begin{enumerate}[noitemsep]
  \item Wie lange dauert es, bis am Rand der Raumstation die Zentrifugalkraft der Erdanziehungskraft entspricht, wenn jedes Triebwerk konstant $\unit[100]{N}$ Schub erzeugt?
  \item Wie groß ist dabei die Winkelbeschleunigung $\alpha$ und wie viele Umdrehungen $n$ macht die Raumstation dabei?
  \item Wie groß ist die Enddrehfrequenz $f_\text{e}$, wenn (ausgehend von $\unit[1]{\frac{U}{min}}$) die Schubkraft sich wie folgt ändert: linearer Anstieg über $\unit[30]{min}$ von $\unit[0]{N}$ auf $\unit[100]{N}$, $\unit[90]{min}$ konstant, dann linearer Abfall auf $\unit[0]{N}$ innerhalb von $\unit[15]{min}$?
\end{enumerate}

\hyperref[lsg:Raumstation]{zur Lösung}

\subsection{Segelboot}
\label{aufg:Segelboot}

Wie muss ein Segel im Wind stehen, wenn Zielkurs und Windrichtung voneinander abweichen?

(So ist die Aufgabenstellung einigermaßen sinnlos. Etwas sinnvoller wird sie, wenn man einen Kiel annimmt.)

\hyperref[lsg:Segelboot]{zur Lösung}


\subsection{Strahlungsdruck}
\label{aufg:Strahlungsdruck}
Ein Laserstrahl der Leistung $P = \unit[19]{mW}$ wird auf eine Kreisfläche vom Radius $r = \unit[6.2]{\mu m}$ fokussiert und trifft dort auf Plastikkugeln der Durchmesser $d = \unit[(0.59, 1.34, 2.68)]{\mu m}$, die in Wasser mit der Viskosität $\eta = \unit[1.05 \cdot 10^{-3}]{\frac{kg}{m\cdot s}}$ schwimmen.
\begin{enumerate}[noitemsep]
  \item Welche Geschwindigkeiten erreichen die Kugeln, wenn man von vollständiger Absorption und Parallelität der Laserstrahlen ausgeht?
  \item Experimentell ergibt sich für die mittlere Kugel eine Geschwindigkeit von $\unit[26]{\frac{\mu m}{s}}$. Begründe die Abweichung von der Rechnung in a)
\end{enumerate}

\hyperref[lsg:Strahlungsdruck]{zur Lösung}


\subsection{Thermische Bewegung von Neutronen}
\label{aufg:ThermischeNeutronen}

Neutronen ($m_\text{n} = \unit[1.675 \cdot 10^{-27}]{kg}$) der Temperatur $T= \unit[4.2]{K}$ treten aus einem Hohlraum aus und bilden hinter Blenden einen horizontalen Strahl, der ein $\unit[200]{m}$ langes Rohr durchquert. Freie Neutronen zerfallen mit einer Halbwertszeit von $t_{1/2} = \unit[10.1]{min}$.
\begin{enumerate}[noitemsep]
  \item Wie groß ist die Geschwindigkeit der Neutronen?
  \item Wie lange und wie tief fallen die Neutronen während der Durchquerung des Rohrs?
  \item Wie viel Prozent der Neutronen zerfallen währenddessen?
\end{enumerate}

\hyperref[lsg:ThermischeNeutronen]{zur Lösung}


\subsection{Rollbewegungen}
\subsubsection{Walze auf schiefer Ebene ("`aus Altklausuren"')}
\label{aufg:WalzeSchiefeEbene}
Gegeben sind ein dünnwandiger und ein Vollzylinder mit ansonsten gleichen Eigenschaften ($r = \unit[3]{cm}$, $m = \unit[2]{kg}$), die eine schiefe Ebene mit Anstellwinkel $\alpha = 30^\circ$ herunterrollen.
\begin{enumerate}[noitemsep]
  \item Trägheitsmomente bezüglich Rollachse?
  \item Beschleunigung der Schwerpunkte?
  \item (?) Rollzeiten bis zur Ruhe?
\end{enumerate}

\hyperref[lsg:WalzeSchiefeEbene]{zur Lösung}

\subsubsection{Rollen auf Ebene}
\label{aufg:WalzeFlach}
Zwei Walzen (eine hohl, andere voll, $m = \unit[20]{kg}$, $r=\unit[20]{cm}$) liegen auf einer Ebene.
\begin{enumerate}[noitemsep]
  \item Welche Beschleunigung erfahren die Zylinderschwerpunkte, wenn die Kraft $F= \unit[40]{N}$ senkrecht zum Boden an der Zylinderachse angreift? Wie groß ist die Haftreibungskraft $F_\text{R}$? Die Zylinder sollen nur rollen.
  \item Wie groß darf die Kraft $F$ höchstens sein, damit die Zylinder nicht anfangen zu gleiten? (Haftreibungskoeffizient $\mu_0 = 0.3$)?
  \item Wie groß ist die gesamte kinetische Energie, der Impuls des Schwerpunktes und der Drehimpuls bei einer Schwerpunktsgeschwindigkeit $v_\text{S} = \unit[10]{\frac{m}{s}}$?
\end{enumerate}

\hyperref[lsg:WalzeFlach]{zur Lösung}

\subsection{Arbeit und Leistung am Wassertank}
\label{aufg:Wassertank}
Betrachte einen zylindrischen Tank mit $r=\unit[2]{m}$ und $h=\unit[3]{m}$.
\begin{enumerate}[noitemsep]
  \item Wie sind Arbeit und Leistung definiert?
  \item Welche Arbeit ist nötig, um den Tank durch eine $\unit[10]{m}$ lange Leitung, die die $\unit[3]{m}$ Höhendifferenz überwindet und oben in den Tank mündet, zu füllen?
  \item Welche Arbeit ist nötig, um den Tank durch eine $\unit[10]{m}$ lange horizontale Leitung, die am Tankboden einmündet, zu befüllen?
  \item Der Tank soll in $\unit[60]{s}$ befüllt werden. Die Pumpe habe einen Wirkungsgrad von $0.95$, der sie antreibende Elektromotor von $0.75$. Berechne die elektrische Arbeit und Leistung des Motors.
\end{enumerate}
Reibungskräfte sind zu vernachlässigen.

\hyperref[lsg:Wassertank]{zur Lösung}

\subsection{Wurf in beschleunigtem Zug}
\label{aufg:WurfZug}
In einem gleichförmig beschleunigtem ($a= \unit[2.23]{\frac{m}{s^2}}$), genügend hohem(!) Zug wird ein Ball unter dem Winkel $\alpha$ zur Horizontalen abgeworfen. Wie groß ist $\alpha$, wenn der Ball an seinem Abwurfort auftreffen soll?

\hyperref[lsg:WurfZug]{zur Lösung}

\subsection{Zeppelin/Heißluftballon}
\subsubsection{Variante 1}
\label{aufg:Zeppelin1}
Wie hoch kann ein Zeppelin mit festem Volumen $V = \unit[25000]{m^3}$ und Heliumfüllung ($\varrho_\text{He} = \unit[0.179]{\frac{kg}{m^3}}$) und weiteren festen Bestandteilen (im Inneren des Volumens) der Masse $m = \unit[16400]{kg}$ bei konstanter Lufttemperatur ($\frac{p_\text{L}}{\varrho_\text{L}} = \text{const.}$) aufsteigen? Für Luft gilt am Boden $p_0 = \unit[100]{\frac{kN}{m^2}}$ und $\varrho_0 = \unit[1.29]{\frac{kg}{m^3}}$.

\hyperref[lsg:Zeppelin1]{zur Lösung}

\subsubsection{Variante 2}
\label{aufg:Zeppelin2}
Wie 1, bis auf: kugelförmiger Ballon mit (konstantem) Durchmesser $d$, Masse $m$. Gesucht sind Kräfte bei Start und maximale Steighöhe für Wasserstoff und Heliumfüllung.

\hyperref[lsg:Zeppelin2]{zur Lösung}

\subsubsection{Variante 3}
\label{aufg:Zeppelin3}
Wie 2, nur hat der Ballon unten eine kleine Öffnung. $d = \unit[3]{m}$, $m=\unit[2]{kg}$. Füllgas Wasserstoff mit $\varrho_\text{H, 0} = \unit[0.009]{\frac{kg}{m^3}}$ oder Helium mit $\varrho_\text{He, 0} = \unit[0.18]{\frac{kg}{m^3}}$.

\hyperref[lsg:Zeppelin3]{zur Lösung}


\subsection{Belasteter Spannungsteiler}
\label{aufg:Spannungsteiler}
Ein Gerät\texttrademark mit Innenwiderstand $R_\text{i}$ wird parallel zu $R_2$, der zusammen mit $R_1$ einen Spannungsteiler der Spannungsquelle $U_0$ bildet, angeschlossen.
\begin{enumerate}[noitemsep]
  \item Bestimmen Sie $U_2(R_\text{i})$, diskutieren und skizzieren Sie diese.
  \item Wie groß ist $U_2$ bei gegebenem $R_1 = \unit[150]{\Omega}$, $R_2 = \unit[70]{\Omega}$, $R_\text{i}=\unit[100]{\Omega}$, $U_0 = \unit[220]{V}$? Wie groß wäre $U_2$, wenn $R_\text{i} \gg R_1$?
  \item Wie groß ist $R_\text{Ges}$? Wie groß sind die Stromstärken durch die einzelnen Widerstände? Welche Leistung steht dem Gerät\texttrademark zur Verfügung?
  \item Welchen Innenwiderstand muss das Gerät\texttrademark haben, damit die Leistung maximal wird? Wie groß sind dann $R_\text{Ges}$, $U_2$, $P_\text{Ges}$?
  \item Vergleichen Sie das $U_2$ mit dem unbelasteten Spannungsteiler. Wie kann das Ergebnis mit dem Schaltbild erklärt werden?
\end{enumerate}

\hyperref[lsg:Spannungsteiler]{zur Lösung}

\subsection{Spulen}
\subsubsection{Ringspule ohne Eisenkern}
\label{aufg:RingspuleOhne}

Eine Ringspule ohne Eisenkern mit dem Ringdurchmesser $D = \unit[30]{cm}$ und dem Wicklungsdurchmesser $d = \unit[3]{cm}$ ist mit $n= 1200$ Windungen Kupferdraht ($\varrho = \unit[1.75 \cdot 10^{-6}]{\Omega \cdot cm}$) vom Querschnitt $A_\text{D} = \unit[0.75]{mm^2}$ bewickelt. Seine Enden liegen an $U = \unit[40]{V}$.
\begin{enumerate}[noitemsep]
  \item Wie groß ist die magnetische Feldstärke im Inneren der Spule, in Abhängigkeit vom Abstand zur Spulenmitte? Wie groß ist die Feldstärke für den kleinsten, mittleren, größten Abstand? Wie groß ist der prozentuale Unterschied zwischen größter und kleinster Feldstärke?
  \item Wie groß ist die Flussdichte auf der Spulenmitte? Wie groß ist der Fluss, wenn überall diese Flussdichte herrschen würde?
  \item Welcher Spannungsstoß wird in einer Sekundärwicklung mit $n_2=5$ Windungen induziert, wenn die Spannung an der Primärwicklung ausgeschaltet wird? (Hinweis: Spannungsstoß ist analog zum Kraftstoß ein Ding der Dimension $\unit{Vs}$.)
\end{enumerate}
\hyperref[lsg:RingspuleOhne]{zur Lösung}

\subsubsection{Ringspule mit Eisenkern und Lücke}
\label{aufg:RingspuleMit}
Die selbe Situation wie in \nameref{aufg:RingspuleOhne}, nur diesmal mit Eisen gefüllter Ring ($\mu_r = 70$).
\begin{enumerate}[noitemsep]
  \item Wie groß sind $H$, $B$, $\Phi$, $L$?
  \item Der Eisenkern wird radial $l_0 = \unit[1]{mm}$ breit aufgesägt. Wie hängen Stromstärke $I$, Flussdichte $B$ und Spaltbreite $l_0$ voneinander ab? Wie groß muss $I$ sein, um dieselbe Flussdichte wie in $a)$ zu erzielen? Wie groß sind die Feldstärken im Eisen und im Spalt?
  \item Mit welcher Kraft ziehen sich die Spaltflächen an?
\end{enumerate}

\hyperref[lsg:RingspuleMit]{zur Lösung}

\subsection{Elektronen im elektrischen/magnetischen Feld}
\label{aufg:ElektronEB}
\begin{enumerate}[noitemsep]
  \item Welche Kraft erfährt ein Elektron im elektrischen Feld?
  \item Welche Geschwindigkeit hat ein Elektron, das durch die Spannung $U$ beschleunigt wurde?
  \item Wie bewegt sich ein Elektron, wenn es mit der Geschwindigkeit $v$ senkrecht zu den Feldlinien auf ein homogenes elektrisches Feld trifft?
  \item Welche Kraft erfährt ein Elektron im magnetischen Feld?
  \item Wie bewegt sich ein Elektron im magnetischen Feld?
  \item Ein Elektron wird durch $U_\text{B}$ beschleunigt und fliegt dann parallel zu den Platten in einen Kondensator mit Länge $l$ und Plattenabstand $d$, an dessen Ende sich ein Schirm befindet. Unter welchem Winkel wird es durch die Spanung $U_\text{A}$ abgelenkt? Wie muss ein Magnetfeld beschaffen sein, dass diese Ablenkung aufhebt?
  \item Rechnung mit Zahlen: gegeben eine Ablenkung von $\alpha = 10^\circ$, $l = \unit[5]{cm}$, $d=\unit[2]{cm}$, $U_\text{A} = \unit[600]{V}$: wie schnell waren die Elektronen vor dem Kondensator? Wie groß muss $B$ sein, um die Ablenkung zu kompensieren?
  \item \todo[inline]{Protonenfilter}

\end{enumerate}

\hyperref[lsg:ElektronEB]{zur Lösung}

\subsection{Energie im Kondensator}
\label{aufg:EnergieKonsensator}
\begin{enumerate}[noitemsep]
  \item Bestimmen Sie die Energie in einem Kondensator
  \item Zu einem geladenen (ansonsten nirgends angeschlossenen) Kondensator wird ein baugleicher parallelgeschaltet. Wie sind nach genügend langer Zeit Ladungen und Energie verteilt?
  \item Vergleichen Sie die Energien und erklären Sie den Unterschied!
\end{enumerate}

\hyperref[lsg:EnergieKondensator]{zur Lösung}

\subsection{Kondensator mit Dielektrika}
\label{aufg:KondensatorDielektrika}
Ein Kondensator mit quadratischen Platten (Länge $l$) und Pattenabstand $d$ ist mit verschiedenen Dielektrika gefüllt ($\varepsilon_1 = 6$, $\varepsilon_2 = 12$).
\begin{enumerate}[noitemsep]
  \item Die Dielektrika liegen nebeneinander zwischen den Platten (je zur Hälfte). Berechnen Sie die Ladungen $Q_1$ und $Q_2$ bei gegebener Gesamtladung $Q$.
  \item Die Dielektrika liegen übereinander (Dicke jeweils $d/2$). Berechnen Sie Teilladungen, Teilspannungen und Teilkapazitäten bei gegebener Gesamtspannung $U$.
\end{enumerate}

\hyperref[lsg:KondensatorDielektrika]{zur Lösung}

\subsection{Widerstände (Stern/Ring)}
\label{aufg:Widerstaende}
\begin{enumerate}[noitemsep]
  \item Berechnen Sie den Gesamtwiderstand zwischen je zwei Punkten in einer Dreiecksschaltung.
  \item Bestimmen Sie die drei Widerstände einer Sternschaltung, so dass zwischen den Eckpunkten die selben Gesamtwiderstände herrschen wie in der Dreiecksschaltung aus a)
\end{enumerate}

\hyperref[lsg:Widerstaende]{zur Lösung}


\subsection{Eis im Kühlschrank}
\label{aufg:Kuehlschrank}
Welche Masse eines Kältemittels (Wirkungsgrad $0.8$, spezifische Verdampfungswärme Ammoniak $\unit[1300]{\frac{kJ}{kg}}$) muss verdampfen, um $\unit[150]{g}$ Wasser von $\unit[16]{^\circ C}$ zu Eis bei $\unit[0]{^\circ C}$ umzuwandeln? Die spezifische Wärmekapazität von Wasser ist $c_\text{W} = \unit[4.182]{\frac{kJ}{kg \cdot K}}$, die spezifische Schmelzwärme von Eis $s_\text{E} = \unit[333.5]{\frac{J}{g}}$.

\hyperref[lsg:Kuehlschrank]{zur Lösung}


\subsection{Schwarzer Strahler/Sonnenscheibe}
\label{aufg:SchwarzeSonne}
  Eine dünne schwarze Scheibe (Radius $r$) steht in der Brennebene eines Parabolspiegels ($R = \unit[10]{cm}$, $f=\unit[1]{m}$) so, dass die Sonne genau auf die Scheibe abgebildet wird. Die Sonne sei ein Schwarzer Strahler bei $T_\text{S} = \unit[6000]{K}$. Berechnen Sie die Temperatur der Scheibe im Gleichgewicht.

  Hinweise: Eliminieren Sie den Sonnenradius $R_\text{S}$ und den Abstand $R_\text{SE}$ zwischen Erde und Sonne durch die Abbildungsgleichung. Der Strahlungsfluss der Scheibe und der Sonne seien proportional zur Fläche ($\approx$ betrachte die Sonne als Scheibe). Die Wärmeleitung durch die Luft wird vernachlässigt.

\hyperref[lsg:SchwarzeSonne]{zur Lösung}


\subsection{Wärmekapazität}
\label{aufg:Waermekap}

Ein Alublock ($l\times b \times h = \unit[5\times 4 \times 2]{cm^3}$, $\varrho = \unit[2.72]{\frac{g}{cm^3}}$, $T_\text{Al} = \unit[100]{^\circ C}$) wird in ein Kalorimeter gegeben, in dem sich Wasser bei $T_\text{W} = \unit[17]{^\circ C}$ befindet. Wie groß ist die spezifische Wärmekapazität $c_\text{al}$, wenn sich eine Gleichgewichtstemperatur $T_\text{GG} = \unit[24.1]{^\circ C}$ einstellt?

Es ist $c_\text{W} = \unit[4.19]{\frac{kJ}{kg \cdot K}}$ und $C_\text{Kal} = \unit[209]{\frac{J}{K}}$.

\hyperref[lsg:Waermekap]{zur Lösung}

\subsection{Wärmeleitung}
\label{aufg:Waermeleitung}
Um wie viel Prozent sinken die Heizkosten, wenn eine $d = \unit[20]{cm}$ dicke Betonwand mit $s = \unit[2]{cm}$ Styropor isoliert wird? Die Wärmeleitkoeffizienten sind $\lambda_\text{B} = \unit[2.1]{\frac{W}{m\cdot K}}$ und $\lambda_\text{S} = \unit[0.3]{\frac{W}{m\cdot K}}$.

% Als Innen- und Außentemperatur schlage ich $T_\text{i} = \unit[20]{^\circ C}$, $T_\text{a} = \unit[8]{^\circ C}$ vor.

\hyperref[lsg:Waermeleitung]{zur Lösung}

\subsection{Dopplereffekt}
\label{aufg:Doppler}
Eine Lok fährt mit $\unit[10]{\frac{km/h}}$ von einer Wand weg und auf einen Beobachter zu. Ein Laserstrahl wird auf der Lok strahlgeteilt und trifft einmal direkt auf den Beobachter und einmal über einen an der Wand befestigten Spiegel. Welche Schwebungsfrequenz wird gemessen?

\hyperref[lsg:Doppler]{zur Lösung}

% ************************************ L O E S U N G E N ************************************ 
\pagebreak
\section{Lösungen}

\subsection{zu \nameref{aufg:Auftrieb}}
\label{lsg:Auftrieb}
\begin{enumerate}[noitemsep]
  \item Durch Boot und Stein wird Wasser entsprechend der Gesamtmasse verdrängt:
  \begin{equation*}
    \Delta V = \frac{m_\text{B} + m_\text{St}}{\varrho_\text{Wasser}} = \frac{\unit[50]{kg}}{\unit[1000]{\frac{kg}{m^3}}} = \unit[0.05]{m^3}.
  \end{equation*}
  Bei der gegebenen Grundfläche $A$ entspricht dies einem Anstieg des Wasserspiegels von
  \begin{equation*}
    \Delta V = \Delta h \cdot A \Rightarrow \Delta h = \frac{\Delta V}{A} = \frac{\unit[0.05]{m^3}}{\unit[25]{m^2}} = \unit[0.002]{m} = \unit[2]{mm}.
  \end{equation*}

  \item Wenn der Stein ins Wasser geworfen wird, verdrängt er nur noch sein Volumen und nicht mehr seine Masse:
\begin{equation*}
  V_\text{St} = \frac{m_\text{St}}{\varrho_\text{St}} = \frac{\unit[10]{kg}}{\unit[3]{\frac{kg}{dm^3}}} = \unit[\frac{10}{3\cdot 1000}]{m^3}.
\end{equation*}
Das Boot verdrängt weiterhin so viel Wasser, wie seinem Gewicht entspricht. Der Wasserspiegel ändert sich gegenüber der Situation "`ohne Boot und Stein"' um
\begin{equation*}
  \Delta h = \frac{V_\text{B} + V_\text{St}}{A} = \frac{\unit[0.04]{m^3} + \unit[\frac{1}{300}]{m^3}}{\unit[25]{m^2}} = \unit[0.0017\overline 3]{m} \approx \unit[1.73]{mm}
\end{equation*}
\end{enumerate}

\subsection{zu \nameref{aufg:Kollergang}}
\label{lsg:Kollergang}
Neben der Gewichtskraft wirkt durch die (erzwungene) Präzession (Aufgabentitel!) eine weitere, senkrecht nach unten gerichtete Kraft.

Die Präzession hängt von der Kreisfrequenz des Rades um seine Hauptträgheitsachse ab. Der Umfang des von dem Rad abgefahrenen Kreises beträgt $U_R = 2\pi R$ und ist wegen $R = 4r$ viermal so groß wie der Umfang des Rades. Daher muss das Rad pro Umdrehung um die senkrechte mittlere Achse vier Umdrehungen um die eigene Achse machen:
\begin{align*}
  \omega_\text{p} & = 2\pi f_\text{P}\\
  \omega_r & = 4 \omega_\text{p}
\end{align*}

Für eine Präzessionsbewegung gilt allgemein
\begin{align*}
  \omega_\text{P} & = \frac{M}{L} = \frac{RF_\text{P}}{L} = \frac{RF_\text{P}}{J\omega_r}\\
  \Rightarrow F_\text{P} & = \frac{\omega_\text{P}\omega_r mr^2}{2R}
\end{align*}

wobei $M$ das Drehmoment auf den Kreisel und $L$ sein Drehimpuls ist. Für einen Zylinder ergibt sich das Trägheitsmoment zu $J = \frac{1}{2}mr^2$

Mit dem Zusammenhang der Kreisfrequenzen $\omega_\text{P}$ und $\omega_r$ ergibt sich schließlich
\begin{equation*}
  F_\text{P} = \frac{16\pi^2f_\text{P}^2mr^2}{2R} = 8\pi^2 \frac{(\unit[0.5]{Hz})^2 \cdot \unit[1500]{kg} \cdot (\unit[0.5]{m})^2}{\unit[2]{m}} = \unit[3701]{N}
\end{equation*}
für die Kraft durch die Präzession. Dazu kommt die Gewichtskraft gemäß $F_\text{G} = mg$:
\begin{equation*}
  F_\text{Ges} = F_\text{P} + F_\text{G} = \unit[3701]{N} + \unit[1500]{kg} \cdot \unit[9.81]{\frac{N}{kg}} = \unit[3701]{N} + \unit[14715]{N} = \unit[18416]{N}
\end{equation*}

\subsection{zu \nameref{aufg:Myonen}}
\label{lsg:Myonen} Das gestrichene System ($s'$, $T'_{1/2}$) ist das erdfeste System, in dem sich die Myonen bewegen.
\begin{enumerate}
  \item Aufgrund der hohen Geschwindigkeit muss die relativistische Zeitdilatation berücksichtigt werden.
  \begin{equation*}
    T'_{1/2} = \frac{T_{1/2}}{\sqrt{1-\frac{v^2}{c^2}}} = \frac{\unit[2\cdot 10^{-6}]{s}}{\sqrt{1-0.9995^2}} = \unit[6.325\cdot 10^{-5}]{s} \quad ( \approx 30 \cdot T_{1/2})
  \end{equation*}
  Die Reichweite ergibt sich aus Geschwindigkeit und durch die Bewegung verlängerte Lebensdauer (die "`innere Uhr"' des Myons ist bewegt, dadurch langsamer):
  \begin{equation*}
    s' = vT'_{1/2} = 0.9995c \cdot \unit[6.325\cdot 10^{-6}]{s} = \unit[1.896 \cdot 10^{4}]{m}
  \end{equation*}

\item Die Reichweite im eigenen Bezugssystem ergibt sich aus der "`tatsächlichen"' mittleren Lebensdauer.
  \begin{equation*}
    s = vT_{1/2} = 0.9995c \cdot \unit[2\cdot 10^{-6}]{s} = \unit[599.29]{m}
  \end{equation*}

\item Die Strecke zwischen Entstehungsort und Erdoberfläche scheint für die Myonen wegen der Längenkontraktion verkürzt.
  \begin{equation*}
    l' = l\sqrt{1-0.9995^2} = \unit[10 \cdot 10^3]{m} \cdot \sqrt{1-0.9995^2} = \unit[316.19]{m}
  \end{equation*}
\item Sowohl a) als auch b) und c) zeigen, dass die Myonen die Erde erreichen können, da die während der mittleren Lebensdauer zurückgelegten Strecken länger als der Abstand Entstehungsort-Erdoberfläche im jeweiligen Bezugssystem ist.
\end{enumerate}

\subsection{zu \nameref{aufg:Raketenstart}}
\label{lsg:Raketenstart}
\begin{enumerate}[noitemsep]
\item Der Impuls des Gasausstoßes muss die Gewichtskraft der Rakete aufbringen:
    \begin{align*}
      F = \dot p = \frac{\text{d}(mv)}{\text{d}t} \stackrel{v=\text{const}}{=} v \frac{\text{d}m}{\text{d}t} \stackrel{!}{=} m_0g \\
      \Rightarrow \frac{\text{d}m}{\text{d}t} = \frac{mg}{v} = \frac{\unit[100 \cdot 10^4]{kg} \cdot \unit[9.81]{\frac{m}{s^2}}}{\unit[4000]{\frac{m}{s}}} = \unit[245.25]{\frac{kg}{s}}
  \end{align*}
\item Die Masse zu einem beliebigen Zeitpunkt beträgt $m(t) = m_0 - n_2t$. Also
  \begin{equation*}
    \frac{1}{2} m_0 = m_0 -n_2t_\text{e} \Rightarrow t_\text{e} = \frac{m_0}{2n_2} = \frac{\unit[100 \cdot 10^3]{kg}}{2 \cdot \unit[500]{\frac{kg}{s}}} = \unit[100]{s}
  \end{equation*}
\item
  \begin{equation*}
    F = m_{1/2}a = v \frac{\text{d}m}{\text{d}t} = \Rightarrow a = \frac{v}{m_{1/2}} \frac{\text{d}m}{\text{d}t} = \frac{\unit[4000]{\frac{m}{s}}}{\unit[50000]{kg}} \cdot \unit[500]{\frac{kg}{s}} = \unit[40]{\frac{m}{s^2}}
  \end{equation*}
  Zusätzlich muss die Erdanziehungskraft berücksichtigt werden, so dass sich insgesamt die Beschleunigung
  \begin{equation*}
    a_\text{Ges} = a - a_{G} = \unit[40]{\frac{m}{s^2}} - \unit[9.81]{\frac{m}{s^2}} = \unit[30.19]{\frac{m}{s^2}}
  \end{equation*}
  ergibt.
\item Mit der Raketengrundgleichung 
  \begin{equation*}
    v(t) =  v_\text{a} \cdot \ln \left(   \frac{m_0}{m_0 - \dot mt}\right)
  \end{equation*}
  ergibt sich zur Zeit, bei der die Rakete ihre halbe Masse verloren hat
  \begin{equation*}
    v(t_\text{e}) = v_\text{a} \ln 2 = \unit[2773]{\frac{m}{s}}.
  \end{equation*}
  Auch hier muss mit der Erdbeschleunigung korrigiert werden, die die ganze Zeit über konstant wirkt:
  \begin{equation*}
    v_\text{G}(t_\text{e}) = v(t_\text{e}) - gt_\text{e} = \unit[2773]{\frac{m}{s}} - \unit[9.81]{\frac{m}{s^2}}\cdot\unit[100]{s} = \unit[1792]{\frac{m}{s}}
  \end{equation*}
\item Wegen $x(t) = \int_0^t v(t')\text{d}t'$ muss die Raketengrundgleichung integriert werden. Analog zur Herleitung der Stammfunktion des Logarithmus ($\int \ln x \text{d}x = x \ln x - x$) kann dazu partielle Integration ($\int f'(x)g(x)\text{d}x = f(x)g(x) - \int f(x)g'(x)\text{d}x$) mit $f' = 1$ angewendet werden, alternativ kann auch
  \begin{equation*}
    \int \ln \frac{a}{a-bt}\text{d}t = \int \ln a \text{d}t - \int \ln (a-bt)\text{d}t
  \end{equation*}
  gefolgt von Integration durch Substitution angesetzt werden.

  \begin{align*}
    \int \ln \frac{a}{a-bt}\text{d}t &= \int \underbrace{1}_{f'} \cdot \underbrace{\ln \frac{a}{a-bt}}_{g}\text{d}t \\
    &= t \cdot \ln \frac{a}{a-bt} - \int t \left( \ln \frac{a}{a-bt} \right)' \text{d}t \\
    &= t \cdot \ln \frac{a}{a-bt} - \int t \frac{a-bt}{a} \cdot \frac{ab}{(a-bt)^2} \text{d}t\\
    &= t \cdot \ln \frac{a}{a-bt} - \int \frac{bt}{a-bt}\text{d}t & | +a -a \\
    &= t \cdot \ln \frac{a}{a-bt} + \int \frac{a-bt-a}{a-bt}\text{d}t\\
    &= t \cdot \ln \frac{a}{a-bt} + t -a \int \frac{1}{a-bt}\text{d}t &|~ \text{Subst.}, u(t) = bt\\
    &= t \cdot \ln \frac{a}{a-bt} + t - \frac{a}{b} \ln \frac{1}{a-bt} + C &|~\text{ab hier optional}\\
    &= t \cdot \ln \frac{a}{a-bt} + t - \frac{a}{b} \ln \frac{a}{a-bt} + C'\\
    &= \left(t-\frac{a}{b}\right) \cdot \ln \frac{a}{a-bt} + t + C'\\
    &= \left(\frac{a-bt}{b}\right) \cdot \ln \frac{a-bt}{a} + t + C'\\
  \end{align*}
Also ist 
\begin{align*}
 x(t) = \int_0^t v(t') \text{d}t' & = v_\text{a} \left[ \frac{m_0 - \dot mt'}{\dot m} \cdot \ln \frac{m_0 - \dot mt'}{m_0} + t' \right]_0^t - \left[ \frac{1}{2}gt'^2\right]_0^t \\
  & = v_\text{a} \left( \frac{m_0 - \dot mt'}{\dot m} \cdot \ln \frac{m_0 - \dot mt'}{m_0} + t' \right) - \frac{1}{2} gt^2
\end{align*}
Was sich für $t = t_\text{e}$ mit $m_0 - \dot m t_\text{e} = m_0/2$ weiter vereinfacht zu
\begin{align*}
  x(t_\text{e}) & = v_\text{a} \left( \frac{m_0}{2\dot m} \cdot \ln \frac{1}{2} + t_\text{e} \right) - \frac{1}{2}gt_\text{e}^2 \\
  &=\unit[4000]{\frac{m}{s}} \left( \frac{\unit[100 \cdot 10^3]{kg}}{2 \cdot \unit[500]{kg}} \cdot \ln \frac{1}{2} + \unit[100]{s} \right) - \frac{1}{2} \cdot \unit[9.81]{\frac{m}{s^2}} \cdot (\unit[100]{s})^2 \\
  &= \unit[73.69]{km}
\end{align*}

\item Aus der Raketengrundgleichung ergibt sich
  \begin{align*}
    v(t) = v_\text{a} \ln \left( \frac{m_0}{m_0-\dot mt} \right) & \Rightarrow e^{\frac{v}{v_\text{a}}} = \frac{m_0}{m_0-\dot mt} \\
    & \Rightarrow e^{-\frac{v}{v_\text{a}}} =  1- \frac{\dot mt}{m_0} \\
    & \Rightarrow t = \frac{m_0}{\dot m}\left(1-e^{-\frac{v}{v_\text{a}}}\right)
  \end{align*}
  also für $v_1 = \unit[7.91]{\frac{km}{s}}$ die Zeit $t_1 = \unit[172.3]{s}$ und für $v_2 = \unit[11.18]{\frac{km}{s}}$ die Zeit $t_2 = \unit[187.8]{s}$. Die Massen ergeben sich mit $m(t) = m_0 - \dot mt$ zu $m_1 = \unit[13.84]{t}$ und $m_2 = \unit[6.11]{t}$.
\end{enumerate}

\subsection{zur \nameref{aufg:Relativitaetstheorie}}
\label{lsg:Relativitaetstheorie}
\begin{enumerate}[noitemsep]
  \item
    \begin{equation*}
      F = ma \Rightarrow a = \frac{F}{m} = \frac{\unit[10]{N}}{\unit[1 \cdot 10^{-3}]{kg}} = \unit[10 \cdot 10^3]{\frac{m}{s^2}}
    \end{equation*}
  \item Bei hohen Geschwindigkeiten bewirkt eine Kraft sowohl eine Geschwindigkeits- als auch eine Massenänderung, daher gilt nicht länger $F = ma$, sondern folgendes:
    \begin{align*}
      F = \frac{\text{d}p}{\text{d}t} = \frac{\text{d}(mv)}{\text{d}t} = v\frac{\text{d}m}{\text{d}t} + m\frac{\text{d}v}{\text{d}t},
    \end{align*}
    wobei
    \begin{align*}
      m = m_0\gamma \text{ mit } \gamma = \frac{1}{\sqrt{1-\frac{v^2}{c^2}}}.
    \end{align*} Also
    \begin{align*}
      F & = v m_0 \frac{\text{d}\gamma}{\text{d}t} + m_0\gamma \frac{\text{d}v}{\text{d}t}\\
      & = m_0\left( v \left( \frac{\text{d}}{\text{d}t} \left( 1-\frac{v^2(t)}{c^2} \right)^{-1/2}\right) +  \left( 1-\frac{v^2(t)}{c^2} \right)^{-1/2} \frac{\text{d}v}{\text{d}t}\right)\\
      &= m_0 \left( v \cdot -\frac{1}{c^2} \cdot a \cdot 2v \cdot -\frac{1}{2} \left( 1-\frac{v^2}{c^2} \right)^{-3/2} + \left( 1-\frac{v^2(t)}{c^2} \right)^{-1/2} a\right)\\
      &= m_0 a \gamma \left(  \frac{v^2}{c^2} \left( 1-\frac{v^2}{c^2} \right)^{-1} + 1\right)\\
      &= m_0a\gamma \left( \frac{v^2}{c^2-v^2} + 1 \right)\\
      &= m_0a\gamma \left( \frac{c^2}{c^2-v^2} \right)\\
      &= m_0a\gamma \left( \frac{1}{\frac{c^2}{c^2} - \frac{v^2}{c^2}} \right)\\
      &= m_0a\gamma \left( \frac{1}{1 - \frac{v^2}{c^2}} \right)\\
      &= m_0a\gamma^3,
    \end{align*}
    für die Beschleunigung also
    \begin{align*}
      a = \frac{F}{m_0\gamma^3} = \frac{\unit[10]{N}\cdot \sqrt{1-\frac{9}{16}}^3}{\unit[0.001]{kg}} = \unit[2894]{\frac{m}{s^2}}
    \end{align*}

\end{enumerate}


\subsection{zu \nameref{aufg:Australien}}
\label{lsg:Australien}
Zentral ist die Tatsache, dass nur die Masse innerhalb der Kugel mit Radius Erdmittelpunkt-momentaner Ort des fallenden Körpers relevant ist.

Für die Gravitationskraft gilt allgemein
\begin{equation*}
  F_\text{G}(r) = - \frac{\gamma mM}{r^2}.
\end{equation*}

Die anziehende Masse $M$ ist ebenfalls von $r$ abhängig:
\begin{equation*}
  M = \varrho V = \varrho \frac{4}{3} \pi  r^3,
\end{equation*}
so dass die wirkende Gravitationskraft 
\begin{equation*}
  F_\text{G}(r) = - \frac{4}{3} \pi \gamma \varrho r \cdot m
\end{equation*}
ist. Mit dem zweiten Newtonschen Gesetz $F = ma = m \ddot r$ ergibt sich die Differentialgleichung
\begin{equation*}
  - \frac{4}{3} \pi \gamma \varrho r = \ddot x,
\end{equation*}
die Bewegung ist also von der Masse des fallenden Körpers unabhängig. Gelöst wird die Gleichung mit dem Ansatz $r(t) = R \cos(\omega t)$ mit $\omega = \sqrt{\frac{4}{3}\pi\gamma\varrho}$. Der Kosinus ergibt sich aus der Anfangsbedingung $\dot r(0) = 0$, die Skalierung mit $R$ wegen $r(0) = R$.
\begin{enumerate}[noitemsep]
  \item Die "`Reisedauer"' ergibt sich aus $\omega$ wie folgt:
    \begin{equation*}
      \omega = 2\pi f = \frac{2\pi}{T} \Rightarrow T = \frac{2\pi}{\omega} = \frac{2\pi}{\sqrt{\frac{4}{3}\pi \cdot \unit[6.674 \cdot 10^{-11}]{\frac{m^3}{kg \cdot s^2}} \cdot \unit[5515]{\frac{kg}{m^3}}}} = \unit[5060]{s}
    \end{equation*}
    für eine "`Rundreise"', also für einen Weg $T_{1/2} = \unit[2530]{s} = \unit[42]{min}\unit[10]{s}$.
  \item Die Höchstgeschwindigkeit wird im Erdmittelpunkt erreicht. Sie ergibt sich aus der Ableitung $\dot r(t) = - R\omega \sin(\omega t)$ mit dem Maximalbetrag $\dot r_\text{max} = R\omega$.
    \begin{equation*}
      \dot r_\text{max} = \unit[6365\cdot10^3]{m} \cdot \sqrt{\frac{4}{3}\pi \cdot \unit[6.674 \cdot 10^{-11}]{\frac{m^3}{kg \cdot s^2}} \cdot \unit[5515]{\frac{kg}{m^3}}} = \unit[7903]{\frac{m}{s}}
    \end{equation*}
  \item Die Gleichung $R\cos(\omega t) = R - v_\text{S}t$ ist nur numerisch zu lösen. Ein TI-83 liefert $t = \unit[69.34]{s}$ und $r = \unit[6341]{km}$.

    Wenn man die Maximalgeschwindigkeit mit der Schallgeschwindigkeit vergleich, so kann man zu dem Schluss kommen, dass der gesuchte Zeitpunkt kurz nach dem Start liegt und eine Taylorentwicklung um $t=0$ der Bewegungsgleichung ansetzen:

    \begin{align*}
      \mathcal R(t)_{t\approx 0} & = R\left(\frac{r(0)}{0!} t^0 + \frac{\dot r(0)}{1!} t^1 + \frac{\ddot r(t)}{2!}t^2\right)\\
      & = R\left(1 + 0 + \frac{-\omega^2}{2}t^2\right)\\
      & = R\left(1 - \frac{\omega^2 t^2}{2}\right)
    \end{align*}
    Dadurch ergibt sich die lösbare quadratische Gleichung \todo{einfach durch t teilen!}
    \begin{align*}
      R\left( 1-\frac{\omega^2 t^2}{2} \right) & = R - v_\text{S}t \\
      \Leftrightarrow R \frac{\omega^2 t^2}{2} & = v_\text{S}t \\
      \Leftrightarrow t^2 - \frac{2v_\text{S}}{R \omega^2}t & = 0 \\
      \Rightarrow t & = - \frac{v_\text{S}}{R\omega^2} \pm \sqrt{\left( \frac{v_\text{S}}{R\omega^2} \right)^2} \\
      \Rightarrow t = 0 \lor t & =\frac{2v_\text{S}}{R\omega^2} = \frac{2 \cdot \unit[340]{\frac{m}{s}}}{\unit[6365 \cdot 10^3]{m} \cdot \left(\unit[1.242 \cdot 10^{-3}]{\frac{1}{s}}\right)^2} = \unit[69.26]{s}
    \end{align*}
    in sehr guter Übereinstimmung mit dem numerisch gefundenem Wert.\todo{Fallstrecke!}
\end{enumerate}

\subsection{zu \nameref{aufg:Raumstation}}
\label{lsg:Raumstation}
\begin{enumerate}[noitemsep]
  \item Gesucht ist $F_\text{Z} = F_\text{G}$, also
    \begin{equation*}
      m\omega r^2 = mg \Rightarrow \omega = \sqrt{\frac{g}{r}} = \sqrt{\frac{\unit[9.81]{\frac{m}{s^2}}}{\unit[50]{m}}} = \unit[0.443]{\frac{1}{s}}
    \end{equation*}
    Die Dauer bis zu dieser Winkelgeschwindigkeit ergibt sich aus $\omega = \alpha t+\omega_0$, also mit $\alpha$ aus b):
    \begin{equation*}
      t = \frac{\omega - \omega_0}{\alpha} = \frac{\unit[0.443]{\frac{1}{s}} - \frac{2\pi}{\unit[60]{s}}}{\unit[5\cdot10^{-5}]{\frac{1}{s^2}}} = \unit[6766]{s}
    \end{equation*}
  \item Die Winkelbeschleunigung hängt von Drehmoment und Trägheitsmoment ab, $M = J\alpha$ (analog $F = ma$). Mit $\vec M = 4 \vec r \times \vec F = 4rF = \unit[20]{kN}$ ergibt sich
    \begin{equation*}
      \alpha = \frac{M}{J} = \frac{\unit[20 \cdot 10^3]{N}}{\unit[0.4 \cdot 10^9]{kg \cdot m^2}} = \unit[5 \cdot 10^{-5}]{\frac{1}{s^2}}
    \end{equation*}
    Die Anzahl der Umdrehungen ergibt sich aus dem Drehwinkel $\varphi = \frac{1}{2}\alpha t^2 + \omega_0t$ (vgl. $s = \frac{1}{2}at^2 + v_0t (+s_0)$ zu 
    \begin{equation*}
      \varphi = \frac{1}{2} \cdot \unit[5\cdot 10^{-5}]{\frac{1}{s^2}} \cdot (\unit[6766]{s})^2 + \unit[\frac{2\pi}{60}]{\frac{1}{s}}\cdot \unit[6766]{s} = 1853
    \end{equation*}
    und daraus die Anzahl der Umdrehungen $n = \varphi / 2\pi = 294.9$.
  \item Der Kraftstoß durch jedes Triebwerk ergibt sich durch Integration $\int F \text{d}t = \frac{1}{2} \cdot \unit[30]{min} \cdot \unit[60]{\frac{s}{min}} \cdot \unit[100]{N} + \unit[90 \cdot 60 \cdot 100]{Ns} + \unit[\frac{1}{2} \cdot 15 \cdot 60 \cdot 100]{Ns} = \unit[675 \cdot 10^3]{Ns}$. Daraus ergibt sich eine Änderung des Drehimpulses $\Delta L = 4r \int F\text{d}t = \unit[135 \cdot 10^6]{Nms}$, was einer Winkelgeschwindigkeitsänderung gemäß $\Delta L = J \Delta \omega$ von
    \begin{equation*}
      \Delta \omega = \frac{\unit[135\cdot 10^6]{Nms}}{\unit[0.4 \cdot 10^9]{kg \cdot m^2}} = \unit[0.3375]{\frac{1}{s}}
    \end{equation*}
    entspricht. Ausgehend von $\omega_0$ also $n_2 = (\omega_0 + \Delta \omega) /2\pi = \unit[4.223]{\frac{1}{min}}$
\end{enumerate}

\subsection{zu \nameref{aufg:Segelboot}}
\label{lsg:Segelboot}
Das Segel kann nur Impuls senkrecht zur Segelfläche aufnehmen, daher muss das Segel in Kursrichtung stehen.

\subsection{zu \nameref{aufg:Strahlungsdruck}}
\label{lsg:Strahlungsdruck}

Die Bestrahlungsstärke im Fokus ergibt sich zu
\begin{equation*}
  E_\text{e, F} =\frac{P}{A}= \frac{P }{\pi r^2} = \frac{\unit[19 \cdot 10^{-3}]{W}}{\pi\cdot (\unit[6.2\cdot 10^{-6}]{m})^2} = \unit[1.573 \cdot 10^8]{\frac{W}{m^2}}
\end{equation*}
Für den Strahlungsdruck gilt $p = \frac{E_\text{e, F}}{c} = \unit[0.5247]{\frac{N}{m^2}}$. Dieser Druck wird von der Querschnittsfläche der Kugeln aufgenommen und führt so zu einer Kraft. Durch die geschwindigkeitsabhängige Stokessche Reibung erreichen die Kugeln so eine konstante Endgeschwindigkeit:
\begin{align*}
  F_\text{S} = p\cdot A_\text{K}~,\quad F_\text{R}  = 6 \pi \eta r v  \\
  \Rightarrow p \cdot A_\text{K} = 6\pi \eta r v \\
  \Rightarrow v  = \frac{p \cdot \pi r^2}{6 \pi r \eta} \\
  \Rightarrow v  = \frac{p\cdot d}{12 \eta}
\end{align*}

und damit $v = \unit[(24.57, 55.80, 111.6)]{\frac{\mu m}{s}}$.

Die Abweichung des experimentellen Wertes kann verschiedene Ursachen haben:
\begin{itemize}[noitemsep]
  \item Die Querschnittsfläche der Kugel steht nicht senkrecht zur Strahlrichtung
  \item Vollständige Absorption ist nicht möglich
  \item Verlust im Wasser
\end{itemize}

\subsection{zu \nameref{aufg:ThermischeNeutronen}}
\label{lsg:ThermischeNeutronen}

\begin{enumerate}[noitemsep]
  \item Die Neutronen sind als ideale Gasteilchen zu betrachten. Gemäß der kinetischen Gastheorie gilt $pV = k_\text{B}T = \frac{1}{3}m_\text{n}\overline v^2$, damit
\begin{equation*}
  \overline v = \sqrt{3 k_\text{B} T \cdot \frac{1}{m_\text{n}}} = \sqrt{3 \cdot \unit[1.381 \cdot 10^{-23}]{\frac{J}{K}} \cdot \unit[4.2]{K} \cdot \frac{1}{\unit[1.675 \cdot 10^{-27}]{kg}}} = \unit[322.3]{\frac{m}{s}}
\end{equation*}
Für die Durchquerung des Rohrs benötigen die Neutronen also $t = s/v = \unit[200]{m} / \unit[322.3]{\frac{m}{s}} = \unit[0.6205]{s}$. Währenddessen fallen sie $h = \frac{1}{2}gt^2 = \frac{1}{2} \cdot \unit[9.81]{\frac{m}{s^2}} \cdot (\unit[0.6205]{s})^2 = \unit[1.889]{m}$ tief.

\item Die Neutronen zerfallen gemäß
  \begin{equation*}
    N(t) = N_0 \cdot 2^{-t/T_{1/2}} = N_0 \cdot 2^{-\unit[0.6205]{s}/\unit[606]{s}} = N_0 \cdot 0.99929
  \end{equation*}
  Es zerfallen also nur $0.071\%$ der Neutronen beim Durchqueren des Rohres.
\end{enumerate}

  
\subsection{Rollbewegungen} 
\subsubsection{zu \nameref{aufg:WalzeSchiefeEbene}}
\label{lsg:WalzeSchiefeEbene}
\begin{enumerate}[noitemsep]
  \item Das Trägheitsmoment eines Hohlzylinders ist $J_\text{H} = mr^2$, das eines Vollzylinders nur halb so groß, $J_\text{V} = \frac{1}{2}mr^2$. Herleitung: für Körper homogener Dichte gilt allgemein $J = \varrho \int r_\bot^2\text{d}V$. ($r_\bot$ ist Abstand des Volumenelements zur Rotationsachse.) In Zylinderkoordinaten:
    \begin{align*}
      J_\text{V} &= \varrho \int_V r_\bot^2\text{d}V \\
      &= \varrho \int_0^h \int _0^{2\pi} \int _0^r r'^2 r' \text{d}r' \text{d}\varphi \text{d}h\\
      &= \varrho \cdot h \cdot 2\pi \cdot \frac{1}{4}r^4 = \frac{1}{2} \varrho\pi r^2 h \cdot r^2 = \frac{1}{2} m r^2
    \end{align*}
    Für den Hohlzylinder gilt wie für die Punktmasse $J_\text{H} = mr^2$, alternativ mit derselben Integration wie oben, wobei allerdings nicht von $0$, sondern von $r-a$ bis $r$ Integriert wird, mit der (kleinen) Wanddicke $a$:
    \begin{align*}
      J_\text{H} & = \varrho \cdot h \cdot 2\pi \cdot \frac{1}{4} \left[ r^4 - (r-a)^4 \right]\\
      &= \varrho \cdot h \cdot \frac{\pi}{2} \left[ r^4 - r^4 + 4r^3a \underbrace{- 6r^2a^2 + 4ra^3 - a^4}_{\approx 0 \text{, weil } a \approx 0} \right] = 2 \pi \varrho h a r^3
    \end{align*}
    Für die Masse des dünnen Zylindermantels ergibt sich $m = 2\pi \varrho har$, daher $J_\text{H} = mr^2$.
\end{enumerate}

\subsection{zu \nameref{aufg:WalzeFlach}}
\label{lsg:WalzeFlach}
\begin{enumerate}[noitemsep]
  \item 
An die Rolle greifen zwei Kräfte an: die beschleunigende Kraft $F$ an der Zylinderachse und die Reibungskraft $F_\text{R}$ (entgegengesetzt) am Auflagepunkt. Zu der Reibungskraft werden im Schwerpunkt $F'_\text{R} = F_\text{R}$ und $-F'_\text{R} = -F_\text{R}$ hinzugefügt, die sich gegenseitig aufheben. Das Kräftepaar ($F_\text{R}$, $-F'_\text{R}$) bewirkt ausschließlich ein Drehmoment $M = rF_\text{R}$, während $F'_\text{R}$ die beschleunigende Kraft vermindert. Es gilt $F - F_\text{R} = ma$ und $M = rF_\text{R} = J\alpha$ und mit der Rollbedingung $a = \alpha r$ ($v = \omega r$, $s = \varphi r$):
\begin{align*}
  F - \frac{Ja}{r^2} = ma \Rightarrow F = \left( m+\frac{J}{r^2} \right)a \Rightarrow a = \frac{F}{m+\frac{J}{r^2}} \\
  F - F_\text{R} = m \frac{F}{m+\frac{J}{r^2}} \Rightarrow F_\text{R} = F \left( 1-\frac{m}{m+\frac{J}{r^2}} \right)
\end{align*}

Für den Hohlzylinder mit $J = mr^2$ ergibt sich also
\begin{align*}
  a = \frac{F}{2m} = \frac{\unit[40]{N}}{2 \cdot \unit[20]{kg}} = \unit[1]{\frac{m}{s^2}} \qquad F_\text{R} = F\left( 1-\frac{1}{2} \right) = \unit[20]{N}
\end{align*}
und für den Vollzylinder mit halbem Trägheitsmoment, $J = \frac{1}{2}mr^2$:
\begin{align*}
  a = \frac{F}{\frac{3}{2}m} = \frac{2 \cdot \unit[40]{N}}{3 \cdot \unit[20]{kg}} = \unit[1.333]{\frac{m}{s^2}} \qquad F_\text{R} = F\left( 1-\frac{2}{3} \right) = \unit[13.33]{N}
\end{align*}

  \item 
    Die (maximale) Haftreibung beträgt $F_\text{R, max} = \mu_0F_\text{N} = \mu_0mg = 0.3 \cdot \unit[20]{kg} \cdot \unit[9.81]{\frac{N}{kg}} = \unit[58.86]{N}$. Für den Hohlzylinder mit $F = 2F_\text{R}$ ist also $F_\text{max} = 2F_\text{R, max} = \unit[117.7]{N}$, für den Vollzylinder $F_\text{max} = 3F_\text{R, max} = \unit[176.6]{N}$.

  \item 
    Der Impuls ist für beide Zylinder gleich, da die Rotation unberücksichtigt bleibt, $p = mv = \unit[20]{kg} \cdot \unit[10]{\frac{m}{s}} = \unit[200]{\frac{kg \cdot m}{s}}$.
    Die kinetische Energie setzt sich aus Rotation und Translation zusammen. Mit $\omega = \frac{v}{r}$ ergibt sich
    \begin{align*}
      E_\text{kin} = E_\text{t} + E_\text{r} = \frac{1}{2}mv^2 + \frac{1}{2}J\omega^2 = \frac{1}{2}v^2 \left( m+\frac{J}{r^2} \right)
    \end{align*}
    also
    \begin{align*}
      E_\text{kin, VZ} = \frac{1}{2} \left(\unit[10]{\frac{m}{s}}\right)^2 \cdot \frac{3}{2}\cdot \unit[20]{kg} = \unit[1.5]{kJ},\quad E_\text{kin, HZ} = \frac{1}{2} \left(\unit[10]{\frac{m}{s}}\right)^2 \cdot 2\cdot \unit[20]{kg} = \unit[2]{kJ}
    \end{align*}
    Die Drehimpulse ergeben sich mit $L = J\omega = J\frac{v}{r}$ zu
    \begin{align*}
      L_\text{VZ} = \frac{1}{2} rmv = \frac{1}{2}\cdot \unit[0.2]{m} \cdot \unit[20]{kg} \cdot \unit[10]{\frac{m}{s}} = \unit[20]{\frac{kg\cdot m^2}{s}}, \quad L_\text{HZ} = 2L_\text{VZ} = \unit[40]{\frac{kg\cdot m}{s}}
    \end{align*}
\end{enumerate}


\subsection{zu \nameref{aufg:Wassertank}}
\label{lsg:Wassertank}
\begin{enumerate}[noitemsep]
  \item $W = Fs$, $P = frac{W}{t}$.
  \item Da das gesamte Wasser um $\Delta h = \unit[3]{m}$ angehoben werden muss, ergibt sich die Hubarbeit gemäß $W_\text{o} = Fs = mgh = V\varrho g h = \pi r^2 h^2 \varrho g = \pi \cdot (\unit[2]{m})^2 \cdot (\unit[3]{m})^2 \cdot \unit[1000]{\frac{kg}{m^3}}\cdot \unit[9.81]{\frac{N}{kg}} = \unit[1.109]{MJ}$.
  \item Um eine bereits im Tank befindliche Wassermenge um $\text{d}h$ anzuheben, ist die Arbeit $\text{d}W = mgh \text{d}h$ aufzubringen. Aufintegriert also
    \begin{align*}
      W_\text{u} &= \int_0^3 \varrho g \pi r^2 h \text{d}h \\
      &= \left[ \frac{1}{2} \varrho g \pi  r^2 h^2 \right]_0^3 = \frac{1}{2} W_\text{o} = \unit[554.7]{kJ}
    \end{align*}
  \item Die mechanische Leistung ist $P = \frac{W}{t} = \frac{\unit[554.7 \cdot 10^3]{J}}{\unit[60]{s}} = \unit[9.246]{kW}$. Durch die Wirkungsgrade erhöht sich die nötige elektrische Leistung auf $P_\text{el} = \frac{1}{0.95 \cdot 0.75} P = \unit[12.98]{kW}$. Die elektrische Arbeit ist dann $W_\text{el} = P_\text{el} \cdot t = \unit[12.98]{kW} \cdot \unit[60]{s} = \unit[778.6]{kJ}$ (jeweils für die effiziente Befüllmethode, für die Befüllung über den oberen Anschluss ergeben sich doppelt so hohe Werte).
\end{enumerate}


\subsection{zu \nameref{aufg:WurfZug}}
\label{lsg:WurfZug}
Im beschleunigtem Bezugssystem Zug wirken neben der Schwerkraft noch die Trägheitskraft aufgrund der Beschleunigung des Bezugssystems. Die Bewegungsgleichungen lauten daher
\begin{align*}
  x(t) &= v_0 \cdot \cos \alpha \cdot t - \frac{1}{2}at^2\\
  y(t) &= v_0 \cdot \sin \alpha \cdot t - \frac{1}{2}gt^2\\
\end{align*}
Der Zeitpunkt, an dem der Ball seine Abfwurfhöhe erreicht ergibt sich aus $y(t) = 0$:
\begin{align*}
  0 = v_0 \cdot \sin \alpha t - \frac{1}{2}gt^2 \Rightarrow t=0 \lor t = \frac{2v_0}{g}\sin\alpha
\end{align*}
Die $x$-Koordinate zu diesem Zeitpunkt muss ebenfalls 0 werden (Rückkehr zum Abwurfort):
\begin{align*}
  v_0 \cdot \cos \alpha \cdot \frac{2v_0}{g}\sin\alpha - \frac{1}{2}a \frac{4v^2}{g^2}\sin^2\alpha = 0 \\
  \Leftrightarrow \frac{2v_0^2}{g}\sin\alpha \left( \cos\alpha - \frac{a}{g}\sin\alpha \right) = 0 \\
  \Rightarrow \cos \alpha = \frac{a}{g}\sin\alpha \\
  \Rightarrow \arctan \alpha = \frac{g}{a}
\end{align*}
Mit den Zahlenwerten $g=\unit[9.81]{\frac{m}{s^2}}$, $a=\unit[2.23]{\frac{m}{s^2}}$ ergibt sich $\alpha = 77.19^\circ$.


\subsection{Zeppelin}
\subsubsection{zu \nameref{aufg:Zeppelin1}}
\label{lsg:Zeppelin1}
Der Luftdruck nimmt gemäß der barometrischen Höhenformel mit zunehmender Höhe ab:
\begin{equation*}
  p_\text{L}(h) = p_0 \cdot e^{-\frac{\varrho_0 hg}{p_0}}
\end{equation*}
und da Druck und Dichte in konstantem Verhältnis stehen, gilt die Gleichung auch für die Dichte:
\begin{equation*}
  \frac{p_\text{L}(h)}{\varrho_\text{L}(h)} = \frac{p_0}{\varrho_0} \Rightarrow \varrho_\text{L}(h) = \frac{\varrho_0}{p_0}p_\text{L}(h) = \varrho_0 \cdot e^{-\frac{\varrho_0 hg}{p_0}}
\end{equation*}

Die Gesamtkraft auf den Zeppelin setzt sich aus Gewichtskraft und Auftrieb zusammen. Damit der Zeppelin nicht (weiter) aufsteigt, muss sie gleich Null sein.
\begin{align*}
  F = \varrho_\text{L}(h) gV - (mg + \varrho_\text{He}gV) & \stackrel{!}{=} 0 \\
  \Rightarrow (\varrho_\text{L}(h) - \varrho_\text{He}) V &= m \\
  \Rightarrow \left( \varrho_0 \cdot  e^{-\frac{\varrho_0gh}{p_0}} -\varrho_\text{He} \right)V &= m\\
  \Rightarrow -\frac{\varrho_0gh}{p_0} &= \ln \left( \frac{\frac{m}{V}+\varrho_\text{He}}{\varrho_0} \right) \\
  \Rightarrow h &= \frac{p_0}{\varrho_0g} \ln \left( \frac{V\varrho_0}{m+V\varrho_\text{He}} \right) = \unit[3.437]{km}
\end{align*}


\subsubsection{zu \nameref{aufg:Zeppelin2}}
\label{lsg:Zeppelin2}
Identisch zu Variante 1, bis auf $V = \frac{4}{3}\pi (d/2)^3$.

\subsubsection{zu \nameref{aufg:Zeppelin3}}
\label{lsg:Zeppelin3}
Durch die Öffnung passt sich der Druck im Ballon stets dem Außendruck an, dadurch verliert der Ballon Gewicht. Diffusion soll wahrscheinlich ausgeschlossen werden.

\begin{align*}
  \varrho_\text{L}(h)gV - (mg + \varrho_\text{H}(h)gV) & \stackrel!= 0 \\
  \Rightarrow (\varrho_\text{L}(h) -\varrho_\text{H}(h))V &= m \\
  \Rightarrow (\varrho_0 \cdot e^{-\frac{\varrho_0gh}{p_0}} - \varrho_\text{H, 0} \cdot e^{-\frac{\varrho_0gh}{p_0}})V &= m \\
  \Rightarrow V(\varrho_0 - \varrho_\text{H, 0}) e^{-\frac{\varrho_0gh}{p_0}} &= m \\
  \Rightarrow h = \frac{p_0}{\varrho_0g} \ln \left( \frac{V(\varrho_0 - \varrho_\text{H, 0})}{m} \right) &= \unit[16.89]{km}
\end{align*}

Für Helium ergibt sich $h = \unit[16.27]{km}$.

\subsection{zu \nameref{aufg:Spannungsteiler}}
\label{lsg:Spannungsteiler}

Die Schaltung besteht aus einer Reihenschaltung von $R_1$ und (einer Parallelschaltung von $R_2$ und $R_\text{i}$). $R_\parallel$ bezeichne den Gesamtwiderstand der Parallelschaltung, dann gilt
\begin{align*}
  U_0 = U_1 + U_2 = R_1I + R_\parallel I \Rightarrow I = \frac{U_0}{R_1 + R_\parallel}
\end{align*}
\begin{enumerate}[noitemsep]
  \item 
Für die Spannung $U_2$ über der Parallelschaltung ergibt sich damit
\begin{align*}
  U_2(R_\text{i}) = R_\parallel I = \frac{1}{\frac{1}{R_2} + \frac{1}{R_\text{i}}} \cdot \frac{U_0}{R_1 + \frac{1}{\frac{1}{R_1} + \frac{1}{R_\text{i}}}} = \dots = \frac{U_0 R_2 R_\text{i}}{R_1R_\text{i} + R_1R_2 + R_2R_\text{i}}
\end{align*}
Kurvendiskussion: Für $R_\text{i} = 0$ ist $U_2 = 0$. Für $R_\text{i} \to \infty$ ergibt sich
\begin{align*}
  \lim_{R_\text{i} \to \infty}U_2 = \lim_{R_\text{i}\to\infty} U_0R_2 \cdot \frac{R_\text{i}}{R_\text{i}}\cdot\frac{1}{R_1+R_2+\frac{R_1R_2}{R_\text{i}}} = \frac{U_0R_2}{R_1+R_2}
\end{align*}
Die Kurve beginnt also im Koordinatenursprung und nähert sich asymptotisch einer Horizontalen. Die Steigung im Ursprung kann bestimmt werden:
\begin{align*}
  \frac{\text{d}U_2}{\text{d}R_\text{i}} &= U_0R_2 \frac{R_1R_\text{i} + R_1R_2 + R_2R_\text{i} - R_i(R_1 + R_2)}{(R_1R_\text{i} + R_1R_2 + R_2R_\text{i})^2}\\
  \frac{\text{d}U_2}{\text{d}R_\text{i}}(R_\text{i}=0) &= \frac{U_0}{R_1}
\end{align*}
\item Durch Einsetzen in die Formel in a) ergibt sich $U_2 = \unit[47.38]{V}$. Wenn $R_\text{i} \gg R_1$ gilt, so befindet man sich in der Situation $R_\text{i} \to\infty$, daher $U_2 = \unit[70]{V}$.
\item Der Gesamtwiderstand ergibt sich aus der Reihen- und Parallelschaltung zu
  \begin{align*}
    R_\text{Ges} = R_1 + R_\parallel = R_1 + \frac{1}{\frac{1}{R_2} + \frac{1}{R_\text{i}}} = R_1 + \frac{R_2R_\text{i}}{R_2 + R_\text{i}} = \unit[191.2]{\Omega}
  \end{align*}
  Die Stromstärken ergeben sich mit dem Ohmschen Gesetz:
  \begin{align*}
    I_1 = \frac{U_0}{R_\text{Ges}(!)} = \unit[1.151]{A}, \quad I_2 = \frac{U_2}{R_2} = \unit[0.6769]{A}, \quad I_\text{i}=\frac{U_2}{R_\text{i}} = \unit[0.4738]{A} = I_1 - I_2
  \end{align*}
  Die Leistung am Gerät\texttrademark ergibt sich mit $P = I_\text{i}^2R_\text{i} = \unit[22.45]{W}$.
\item Um $P_\text{i}$ nach $R_\text{i}$ abzuleiten muss man beachten, dass $I_\text{i}$ ebenfalls von $R_\text{i}$ abhängig ist (siehe c)).
  \begin{align*}
    P_\text{i} & = I_\text{i}^2R_\text{i} = \frac{U_2^2}{R_i} = \frac{(U_0R_2R_\text{i})^2}{R_\text{i}\underbrace{(R_1R_2+R_1R_\text{i}+R_2R_\text{i})}_{=:(\dots)}{}^2} = U_0^2R_2^2 \frac{R_\text{i}}{(\dots)^2}\\ % leeres {} um \underbrace zu beenden (sonst landet die 2 als overbrace über dem underbraceten)
    \Rightarrow \frac{\text{d}P_\text{i}}{\text{d}R_\text{i}} &= U_0^2R_2^2 \frac{(\dots)^2 - R_\text{i}(R_1+R_2) \cdot 2(\dots)}{(\dots)^4} = U_0^2R_2^2 \frac{R_1R_2 - R_1R_\text{i} - R_2R_\text{i}}{(\dots)^3}
  \end{align*}
  Aus der Extremalbedingung $\frac{\text{d}P}{\text{d}R} = 0$ ergibt sich, da der Zähler Null werden muss:
  \begin{align*}
    R_\text{i, max} = \frac{R_1R_2}{R_1+R_2} = \unit[47.73]{\Omega}
  \end{align*}
  Gesamtwiderstand, Spannung am Gerät\texttrademark und Leistung ergeben sich wie in a) bzw. c)
  \begin{align*}
    U_2(R_\text{i, max}) &=  \frac{U_0 R_2 R_\text{i, max}}{R_1R_\text{i, max} + R_1R_2 + R_2R_\text{i, max}} = \unit[35]{V}\\
    R_\text{Ges, max} &=  R_1 + \frac{R_2R_\text{i, max}}{R_2 + R_\text{i, max}} = \unit[178.4]{\Omega}\\
    P_\text{max} &= \frac{U_2^2}{R_\text{i, max}} = \unit[25.67]{W}
  \end{align*}
\item Für den unbelasteten Spannungsteiler gilt $R_\text{i} \to \infty$, also (vgl. b)): $U_2(\infty) = \unit[70]{V} = 2\cdot U_2(R_\text{i, max})$
\end{enumerate}

\subsection{Spulen}
\subsubsection{zu \nameref{aufg:RingspuleOhne}}
\label{lsg:RingspuleOhne}
\begin{enumerate}[noitemsep]
  \item 
Gemäß der 4. Maxwellgleichung gilt für einen Integrationsweg entlang des Rings
\begin{equation*}
  \oint \vec H \cdot \text{d}\vec s = nI
\end{equation*}
und da $\vec H \parallel \vec s$ und der Integrationsweg $\pi D$ lang ist, 
\begin{equation*}
  H = \frac{nI}{\pi D}
\end{equation*}
Die Stromstärke ergibt sich aus Spannung, Drahtgeometrie und spezifischem Widerstand:
\begin{equation*}
  I = \frac{U}{R} = \frac{UA}{l\varrho} = \frac{UA}{n \cdot \pi \cdot d \cdot \varrho} = \frac{\unit[40]{V}\cdot \unit[0.75\cdot 10^{-6}]{m^2}}{1200\pi \cdot \unit[0.03]{m}\cdot \unit[1.75 \cdot 10^{-8}]{\Omega \cdot m}} = \unit[15.16]{A}
\end{equation*}

Damit
\begin{align*}
  H_\text{m} &= \frac{n}{\pi D} \cdot \frac{UA}{n\pi d \varrho} = \frac{UA}{\pi^2\varrho d D} \\
  &= \frac{\unit[40]{V} \cdot \unit[0.75\cdot10^{-6}]{m^2}}{\pi^2 \cdot \unit[1.75\cdot 10^{-8}]{\Omega \cdot m} \cdot\unit[0.03]{m} \cdot \unit[0.3]{m}} \\
  &= \unit[1.93 \cdot 10^4]{\frac{V}{\Omega \cdot m}} = \unit[1.93 \cdot 10^4]{\frac{A}{m}}
\end{align*}
für die Spulenmitte, $H_\text{i} = \unit[2.144\cdot 10^4]{\frac{A}{m}}$ am inneren und $H_\text{a} = \unit[1.754\cdot 10^4]{\frac{A}{m}}$ am äußenren Spulenrand. (NB: unabhängig von der Windungszahl!)
\item Wegen $B = \mu_0 H$ ist in der Spulenmitte die Flussdichte
  \begin{equation*}
    B_\text{m} = 4\pi \cdot \unit[10^-7]{\frac{N}{A^2}} \cdot \unit[1.93 \cdot 10^4]{\frac{A}{m}} = \unit[2.425\cdot 10^{-2}]{\frac{N}{A\cdot m}}= \unit[2.425\cdot 10^{-2}]{T}
  \end{equation*}
  und der bei konstanter Flussdichte angenommene Fluss $\Phi$
  \begin{equation*}
    \Phi = BA = \unit[2.425\cdot 10^{-2}]{T} \cdot 2\pi \cdot (\unit[0.015]{m})^2 = \unit[1.714 \cdot 10^{-5}]{Wb}
  \end{equation*}
\item Durch das Ausschalten der Spannung ändert sich der magnetische Fluss in der Spule, wodurch in der Sekundärspule (mit $n_2$ Windungen) eine Spannung induziert wird. Der Spannungsstoß ist die über die "`Abklingzeit"' aufintegrierte Spannung. Ausgehend von dem Induktionsgesetz ergibt sich
  \begin{align*}
    U_\text{ind} & = -n_2\frac{\text{d}\Phi}{\text{d}t} \\
    \Rightarrow U_\text{ind} \text{d}t &= - n_2 \text{d}\Phi \Rightarrow \int U_\text{ind}\text{d}t = n_2 \Phi
  \end{align*}
\end{enumerate}
Der genaue zeitliche Verlauf des magnetischen Flusses ist unerheblich, es genügt zu wissen, dass er von $\Phi$ auf 0 abnimmt. Zahlenmäßig ergibt sich $\int U\text{d}t = \unit[8.572\cdot 10^{-5}]{Vs}$.

\subsubsection{zu \nameref{aufg:RingspuleMit}}
\label{lsg:RingspuleMit}

\begin{enumerate}[noitemsep]
  \item Die magnetische Feldstärke ist von der Permeabilität unabhängig, daher gilt wie in der \hyperref[lsg:RingspuleOhne]{Lösung ohne Eisenkern} $H_\text{(m, Fe)} = \unit[1.93\cdot 10^4]{\frac{A}{m}}$. Wegen $B = \mu_0 \mu_r H$ ist die Flussdichte im Eisen
    \begin{equation*}
      B_\text{(m, Fe)} = \unit[4\pi \cdot 10^{-7}]{\frac{Vs}{Am}} \cdot 70 \cdot \unit[1.93\cdot 10^4]{\frac{A}{m}} = \unit[1.698]{T}
    \end{equation*}
    und der Fluss
    \begin{equation*}
      \Phi = B_\text{(m, Fe)} \cdot A = \unit[1.698]{T} \cdot \pi (\unit[0.015]{m})^2 = \unit[1.2 \cdot 10^{-3}]{Wb}
    \end{equation*}
    Die Induktivität ergibt sich (mit dem Strom aus \ref{lsg:RingspuleOhne}) zu
    \begin{equation*}
      L = n \frac{\Phi}{I} = 1200 \cdot \frac{\unit[1.2 \cdot 10^{-3}]{Wb}}{\unit[15.16]{A}} = \unit[9.499 \cdot 10^{-2}]{H}
    \end{equation*}

  \item Mit Maxwell 4 ergibt sich $\oint H \text{d}s = nI = H_\text{Fe}l_\text{Fe} + H_\text{L}l_0$, da das Feld nicht konstant ist, sondern in zwei Teile (im Eisen und in Luft) zerfällt. Aus Maxwell 2 ($\oint \vec B \cdot \text{d}\vec A = 0$) ergibt sich, dass $\vec B$ in Eisen und Luft gleich sein müssen (Integration über "`Pillendose"' mit einem Deckel in Luft, anderem in Eisen). Wegen $B = \mu_0(\mu_r)H$ muss also die Feldstärke unterschiedlich sein.
    \begin{align*}
      & \oint H\text{d}s = nI = H_\text{Fe}l_\text{Fe} + H_\text{L}l_0 = \frac{B'}{\mu_0\mu_r}l_\text{Fe} + \frac{B'}{\mu_0}l_0 = \frac{B'}{\mu_0}\left( \frac{l_\text{Fe}}{\mu_r} + l_0 \right)\\
      & \Rightarrow B' = \frac{\mu_0 \cdot n \cdot I}{\left( \frac{l_\text{Fe}}{\mu_r} + l_0 \right)} = \frac{\mu_0 \mu_r nI}{l_\text{Fe} + \mu_rl_0} = \frac{\mu_0\mu_rnI}{l_0(\mu_r -1 ) + 2\pi r} = \unit[1.582]{T}
    \end{align*}
    Um dieselbe Flussdichte wie in a) zu erhalten, muss die Stromstärke erhöht werden:
    \begin{align*}
      B = \frac{\mu_0\mu_r nI}{l_0(\mu_r - 1) + 2\pi r} \Rightarrow I = B \frac{l_0(\mu_r-1)+2\pi r}{\mu_0\mu_rn} \\
      \Rightarrow I = \unit[1.698]{\frac{N}{A\cdot m}} \cdot \frac{\unit[0.001]{m} \cdot (70-1) + 2\pi \cdot \unit[0.15]{m}}{4\pi\cdot\unit[10^{-7}]{\frac{N}{A^2}} \cdot 70 \cdot 1200} = \unit[16.27]{A}
    \end{align*}
    Die Feldstärken in Eisen und Luft ergeben sich aus der Flussdichte $B'$ zu
    \begin{align*}
      H_\text{L} &= \frac{B'}{\mu_0} = \frac{\unit[1.582]{T}}{4\pi\cdot\unit[10^{-7}]{\frac{N}{A^2}}} = \unit[1.259\cdot 10^6]{\frac{A}{m}} \\
      H_\text{Fe} &= \frac{B'}{\mu_0\mu_r} = \frac{\unit[1.582]{T}}{4\pi\cdot\unit[10^{-7}]{\frac{N}{A^2}}\cdot 70 } = \unit[1.798 \cdot 10^4]{\frac{A}{m}}
    \end{align*}
  \item Kräfte können durch Ableitung einer Energie/Arbeit nach dem Weg bestimmt werden (vgl. $F_g = mg$, $E_\text{pot} = mgh$). Der Energieinhalt eines (homogenen) magnetischen Feldes vom Querschnitt $A$ und der Länge $s$ ist (aus $W_\text{m} = \frac{1}{2}LI^2$ und Betrachtung einer langen Spule)
    \begin{equation*}
      W_\text{m} = \frac{1}{2}BHAs
    \end{equation*}
Die Wegableitung liefert also 
\begin{equation*}
  F = \frac{1}{2}HBA=  \frac{1}{2} \Phi  H_\text{L} = \frac{1}{2} \cdot \unit[1.2\cdot 10^{-3}]{Wb} \cdot \unit[1.259 \cdot 10^6]{\frac{A}{m}} = \unit[755.4]{N}
\end{equation*}
\end{enumerate}

\subsection{zu \nameref{aufg:ElektronEB}}
\label{lsg:ElektronEB}
\begin{enumerate}[noitemsep]
  \item $\vec F = e\vec E$ mit der Elementarladung $e$.
  \item Die potentielle Energie durch die Lage im elektrischen Feld, $Ue$, wird in kinetische Energie umgewandelt: $Ue = mv^2/2 \Rightarrow v = \sqrt{2Ue/m}$.
  \item Durch die Überlagerung einer gleichförmigen und einer gleichmäßig beschleunigten Bewegung entsteht völlig analog zum horizontalem Wurf eine Parabel. Mit $a = F/m$ ergibt sich
    \begin{align*}
      x(t) &= vt\\
      y(t) &= \frac{1}{2} \frac{eE}{m_\text{e}} t^2\\
      \Rightarrow y(x) & = \frac{1}{2} \frac{eE}{m_\text{e}v^2}x^2
    \end{align*}
  \item Die Lorentzkraft $\vec F_\text{L} = e(\vec v \times \vec B)$.
  \item Auf einer Kreisbahn, da stets eine Ablenkung rechtwinklig zur Momentangeschwindigkeit stattfindet. Der Radius dieser Bahn ergibt sich aus der nötigen Zentripetalkraft:
    \begin{align*}
      F_\text{L} &= F_\text{Z}\\
      \Rightarrow evB &= \frac{m_\text{e}v^2}{r}\\
      \Rightarrow r &= \frac{m_\text{e}}{e} \frac{1}{B} v
    \end{align*}
  \item In der Parabelgleichung aus c) wird $E$ durch $U_\text{A}/d$ und die Geschwindigkeit $v$ gemäß b) ersetzt:
    \begin{align*}
      y(l) &= \frac{1}{2}\frac{e}{m_\text{e}} \frac{U_\text{A}}{d} \frac{l^2}{v^2}\\
      \Rightarrow y(l) &= \frac{1}{2}\frac{e}{m_\text{e}} \frac{U_\text{A}}{d} \frac{l^2}{v^2}\\
      \Rightarrow y(l) &= \frac{1}{2}\frac{e}{m_\text{e}} \frac{U_\text{A}}{d} \frac{l^2}{{2U_\text{B}e}/{m_\text{e}}} = \frac{1}{4} \frac{U_\text{A}}{dU_\text{B}}l^2\\
    \end{align*}
    Die Winkelabweichung ergibt sich aus dem Tangens:
    \begin{align*}
      \tan \alpha = \frac{y(l)}{l} = \frac{1}{4}\frac{U_\text{A}}{dU_\text{B}}
    \end{align*}
    ACHTUNG: wahrscheinlich ist hier eher der Austrittswinkel aus dem Feld zur Horizontalen gesucht, der sich aus der Richtung der Momentangeschwindigkeit beim Austritt ergibt!

    Für eine gerade Flugbahn müssen sich Kraft durch Magnetfeld und elektrisches Feld gegenseitig aufheben. Aufgrund der geschwindigkeitsabhängigen Lorentzkraft ist dies für genau eine Geschwindigkeit gegeben. Aus $\vec F_\text{L} = -\vec F_\text{E} \Rightarrow \vec v \times \vec B = - \vec E$ ergibt sich, dass ein nach "`rechts"' fliegendes Elektron durch ein "`von vorne nach hinten"' gerichtetes magnetisches und ein "`von oben nach unten"' gerichtetes elektrisches Feld geradeaus weiterfliegen kann. ("`Rechte-Hand-Regel"' für das Kreuzprodukt und dann Umkehrung von $\vec E$, oder direkt "`Linke-Hand-Regel"'). Für die benötigte Flussdichte $B$ gilt 
    \begin{align*}
      |B| = \frac{E}{v} = \frac{U_\text{A}}{d\sqrt{2U_\text{B}e/m_\text{e}}}
    \end{align*}
  \item Ausgehend von den Gleichungen aus f):
    \begin{align*}
      y(l) &= \frac{1}{2}\frac{e}{m_\text{e}} \frac{U_\text{A}}{d} \frac{l^2}{v^2}, \quad \tan\alpha = \frac{y(l)}{l}\\
      \Rightarrow \tan \alpha & =  \frac{1}{2}\frac{e}{m_\text{e}} \frac{U_\text{A}}{d} \frac{l}{v^2}\\
      \Rightarrow v & = \sqrt { \frac{1}{2}\frac{e}{m_\text{e}} \frac{U_\text{A}}{d} \frac{l}{\tan \alpha}} = \unit[2.735 \cdot 10^7]{\frac{m}{s}} \approx 0.1c\\
      |B| & = \frac{U_\text{A}}{dv} = \frac{\unit[600]{V}}{\unit[0.02]{m} \cdot \unit[2.735 \cdot 10^7]{\frac{m}{s}}} = \unit[1.097 \cdot 10^{-3}]{T}
    \end{align*}
\end{enumerate}


\subsection{zu \nameref{aufg:EnergieKonsensator}}
\label{lsg:EnergieKondensator}
\begin{enumerate}[noitemsep]
  \item Die Verschiebungsarbeit, um eine Ladung $\text{d}Q$ bei gegebener Spannung von der "`leeren"' zur "`vollen"' Platte zu verschieben, ist $\text{d}W = U\text{d}Q$. Da sich die Spannung eines Kondensators aus Kapazität und Ladung ergibt, kann nach Einsetzen von $U = Q/C$ integriert werden:
    \begin{align*}
      \int \text{d}W & = \int \frac{Q}{C}\text{d}Q\\
      \Rightarrow W &= \frac{1}{2C}Q^2 = \frac{1}{2}CU^2 = \frac{1}{2}QU
    \end{align*}
  \item Durch den zweiten Kondensator verteilt sich die konstante Ladung $Q$ nun auf die doppelte Kapazität:
    \begin{align*}
      W_1 = \frac{1}{2C}Q^2 \qquad W_2 = \frac{1}{4C}Q^2 = \frac{1}{2}W_1~!
    \end{align*}
  \item Bei exakter Berechnung des Umladevorgangs über einen ohmschen Widerstand ergibt sich, dass genau die Hälfte der ursprünglich gespeicherten Energie im Widerstand gemäß $P = RI^2$ in Wärme umgesetzt wird\\
    \todo[inline]{Rechnung?}
\end{enumerate}

\subsection{zu \nameref{aufg:KondensatorDielektrika}}
\label{lsg:KondensatorDielektrika}
\begin{enumerate}[noitemsep]
  \item Wegen $U = Ed = \text{const}$, $D = \varepsilon_0\varepsilon_\text{i}E$ und $Q = DA$ verteilt sich die Ladung aufgrund der unterschiedlichen $\varepsilon_\text{i}$ nicht gleichmäßig auf den Platten. Die elektrische Flussdichten ergeben sich zu $D_\text{i} = \varepsilon_0\varepsilon_\text{i}E = \varepsilon_0\varepsilon_\text{i}\frac{U}{d}$ und die Gesamtkapazität aus 
    \begin{align*}
      Q &= D_1\frac{A}{2} + D_2\frac{A}{2} = \frac{A}{2} \varepsilon_0 \frac{U}{d}(\varepsilon_1+\varepsilon_2)\\
      Q &= CU\\
      \Rightarrow C & = \frac{\varepsilon_0A}{2d}(\varepsilon_1+\varepsilon_2)\\
      C&=C_0 \frac{\varepsilon_1+\varepsilon_2}{2}
    \end{align*}
    Betrachtet man den Kondensator als Parallelschaltung von zwei Kondensatoren mit unterschiedlichen Dielektrika, so ergeben sich die Teilladungen wie folgt:
    \begin{align}
      U &= \frac{Q_1}{C_1} = \frac{Q_2}{C_2} \label{equ:uqc}\\
      Q &= Q_1+Q_2 \label{equ:qges}\\
      C_\text{i} &= \frac{\varepsilon_0\varepsilon_\text{i}A}{2d} \label{equ:ci}
    \end{align}
    Aus (\ref{equ:uqc}) und (\ref{equ:qges}) ergibt sich
    \begin{align*}
      Q_2 = Q \frac{C_2}{C_1+C_2}
    \end{align*} und daraus mit (\ref{equ:ci})
    \begin{align*}
      Q_2 = Q \frac{ \frac{ \varepsilon_0 \varepsilon_2A} {2d}} {\frac{ \varepsilon_0\varepsilon_1A + \varepsilon_0\varepsilon_2A} {2d}} = Q\frac{\varepsilon_2}{\varepsilon_1+\varepsilon_2} = Q \frac{12}{6+12} = \frac{2}{3}Q
    \end{align*}
    und analog
    \begin{align*}
      Q_1 = \frac{\varepsilon_1}{\varepsilon_1+\varepsilon_2} = Q\frac{6}{6+12} = \frac{1}{3}Q
    \end{align*}
  \item Der Kondensator mit parallel zu den Platten unterteiltem Dielektrikum verhält sich wie eine Parallelschaltung von Kondensatoren. Wegen $Q=DA$ muss $D$ überall im Kondensator konstant sein, das $E$-Feld hat an dem Übergang zwischen den Dielektrika einen Sprung, und damit auch der Spannungsverlauf. Das Problem entspricht zwei in Reihe geschalteten Kondensatoren, bei denen sich auf den mittleren beiden Platten Ladungen nur durch Influenz verschieben, so dass auf allen vier Platten betragsmäßig gleiche Ladungen vorliegen.
    Die Gesamtkapazität ergibt sich aus
    \begin{align*}
      U &= \frac{Q}{C_1} + \frac{Q}{C_2}  = \frac{Q}{C_\text{ges}}\\
      \Leftrightarrow \frac{U}{Q} &= \underbrace{\frac{1}{C_1} + \frac{1}{C_2}  = \frac{1}{C_\text{ges}}}\\
      \Rightarrow C_\text{ges} &= \frac{1}{\frac{1}{C_1}+\frac{1}{C_2}} = \frac{1}{\frac{d}{2\varepsilon_0\varepsilon_1A} + \frac{d}{2\varepsilon_0 \varepsilon_2 A}} = \frac{2 \varepsilon_0 \varepsilon_1 \varepsilon_2 A}{\varepsilon_2 d + \varepsilon_1 d} = 2 \frac{\varepsilon_0 A}{d} \frac{\varepsilon_1 \varepsilon_2}{\varepsilon_1 + \varepsilon_2}
    \end{align*}
    Die Spannung in einem Kondensator findet man über
    \begin{align*}
      Q &= U C_\text{ges} \\
      U_1 &= \frac{Q}{C_1} = U \frac{C_\text{ges}}{C_1} = U \frac{ 2 \frac{\varepsilon_0 A}{d} \frac{\varepsilon_1 \varepsilon_2}{\varepsilon_1 + \varepsilon_2}}{2\frac{\varepsilon_0 A}{d} \varepsilon_1} = U\frac{\varepsilon_2}{\varepsilon_1 + \varepsilon_2} = \frac{2}{3}U\\
      U_2 &= U \frac{\varepsilon_1}{\varepsilon_1 + \varepsilon_2} = \frac{1}{3}U
    \end{align*}

\end{enumerate}




\subsection{zu \nameref{aufg:Widerstaende}}
\label{lsg:Widerstaende}
Benennung: $R_1$ zwischen $A$ und $B$, $R_2$ zwischen $B$ und $C$ im Dreieck; $R_\text{a}$ an $A$ im Stern.

\begin{enumerate}[noitemsep]
  \item Der Gesamtwiderstand zwischen $A$ und $B$ bestimmt sich aus Parallelschaltung von $R_1$ und (der Reihenschaltung von $R_2$ und $R_3$), daher
    \begin{equation*}
      R_\text{AB} = \frac{1}{\frac{1}{R_1} + \frac{1}{R_2+R_3}} = \frac{1}{\frac{R_2+R_3}{R_1(R_2+R_3)} + \frac{R_1}{R_1(R_2+R_3)}} = \frac{R_1(R_2+R_3)}{R_1+R_2+R_3},
    \end{equation*}
    also "`direkt $\cdot$ Umweg / alle"'.
  \item Es ergibt sich das Gleichungssystem
    \begin{align}
      R_\text{a}+R_\text{b} = R_\text{AB} = \frac{R_1(R_2+R_3)}{R_1+R_2+R_3}\label{equ:ab}\\
      R_\text{a}+R_\text{c} = R_\text{AC} = \frac{R_3(R_1+R_2)}{R_1+R_2+R_3}\label{equ:ac}\\
      R_\text{b}+R_\text{c} = R_\text{BC} = \frac{R_2(R_1+R_3)}{R_1+R_2+R_3}\label{equ:bc}
    \end{align}
    in dem durch $((\ref{equ:ab}) - (\ref{equ:ac})) + (\ref{equ:bc})$ die Variable $R_\text{b}$ isoliert werden kann:
    \begin{align*}
      ((R_\text{a} + R_\text{b}) - (R_\text{a} + R_\text{c})) + (R_\text{b} + R_\text{c}) &= \frac{R_1(R_2+R_3) - R_3(R_1+R_2) + R_2(R_1+R_3)}{R_1+R_2+R_3}\\
      \Rightarrow 2R_\text{b} & = \frac{R_1R_2 + R_1R_3 - R_1R_3 - R_2R_3 + R_1R_2 + R_2R_3}{R_1+R_2+R_3}\\
      \Rightarrow 2R_\text{b} & = \frac{2R_1R_2}{R_1+R_2+R_3}\\
      \Rightarrow R_\text{b} &= \frac{R_1R_2}{R_1+R_2+R_3},
    \end{align*}
    also "`Produkt der (im Dreieck) angeschlossenen / alle"'
\end{enumerate}


\subsection{zu \nameref{aufg:Kuehlschrank}}
\label{lsg:Kuehlschrank}
Dem Wasser muss Wärme zur Abkühlung auf $\unit[0]{^\circ C}$ und dann zum Erstarren entzogen werden.
\begin{align*}
  \Delta Q_\text{W} &= C_\text{W}\Delta T + m_\text{W}s_\text{E}\\
  & = m_\text{w} c_\text{W}\Delta T + m_\text{W}s_\text{E}
\end{align*}
Das Kältemittel entzieht beim Verdampfen der Umgebug Wärme entsprechend der Verdampfungswärme, mit Berücksichtigung des Wirkungsgrades:
\begin{align*}
  \Delta Q_\text{KM} & = \eta m_\text{KM}v_\text{KM}\\
  \Rightarrow m_\text{KM} & = \frac{m_\text{W}c_\text{W} \Delta T + m_\text{W}s_\text{E}}{\eta v_\text{KM}} \\
  & = \frac{\unit[0.15]{kg} \cdot \unit[4182]{\frac{J}{kg \cdot K}} \cdot \unit[16]{K} + \unit[0.15]{kg}\cdot\unit[333.5 \cdot 10^3]{\frac{J}{kg}}}{ 0.8 \cdot \unit[1.3 \cdot 10^6]{\frac{J}{kg}}} = \unit[57.75]{g}
\end{align*}


\subsection{zu \nameref{aufg:SchwarzeSonne}}
\label{lsg:SchwarzeSonne}
Die Abbildungsgleichung führt zu
\begin{align*}
  \frac{B}{G} = \frac{b}{g} = \frac{f}{g} \Rightarrow f = \frac{Bg}{G}
\end{align*}
Die von der Sonnenscheibe abgestrahlte Leistung ergibt sich aus dem Stefan-Boltzmann-Gesetz:
\begin{align*}
  P_\text{S} = 4\pi R_\text{S}^2 \cdot \sigma T_\text{S}^4
\end{align*}
Die Leistung, die auf dem Spiegel ankommt, ergibt sich aus dem Verhältnis von Spiegelfläche zu der Kugeloberfläche mit Radius Erde-Sonne:
\begin{align*}
  P_\text{Sp} = P_\text{S} \cdot \frac{\pi R^2}{4\pi R_\text{SE}^2} = \frac{4\pi R_\text{S}^2 \cdot \sigma T_\text{S}^4 \cdot \pi R^2}{4\pi R_\text{SE}^2} = \pi \sigma T_\text{S}^4 \frac{R^2 R_\text{S}^2}{R_\text{SE}^2}
\end{align*}
  Die Temperatur der Scheibe entsteht durch die gebündelte Strahlungsleistung des Spiegels aus dem nach $T$ umgestellten Stefan-Boltzmann-Gesetz. Da die Scheibe auf beiden Seiten strahlt, kommt ein Faktor 2 ins Spiel:
  \begin{align*}
    T_\text{Sch} = \sqrt[4]{\frac{P_\text{Sp}}{2 \sigma A_\text{Sch}}} = \sqrt[4]{\frac{\pi \sigma T_\text{S}^4 R^2 R_\text{S}^2}{R_\text{SE}^2 \cdot 2 \sigma \pi r^2}} = T_\text{S} \sqrt{\frac{R}{2} \cdot \frac{R_\text{S}}{r R_\text{SE}}}
  \end{align*}
  Wie man leicht sieht, steht unter der Wurzel der Kehrwert der Brennweite, so dass sich insgesamt ergibt
  \begin{align*}
    T_\text{Sch} = T_\text{S} \sqrt{\frac{R}{2f}} = \unit[6000]{K} \cdot \sqrt{\frac{ \unit[0.1]{m}}{2 \cdot \unit[1]{m}}} = \unit[1341]{K}
  \end{align*}

\subsection{zu \nameref{aufg:Waermekap}}
\label{lsg:Waermekap}
Nötige Annahme ist, dass Wasser und Kalorimeter auf derselben Anfangstemperatur sind. Im Gleichgewicht ist die gesamte Wärme auf die gesamte Masse verteilt, also
\begin{align*}
  Q_\text{Ges} & = T_\text{Al}C_\text{Al} + T_\text{W}(C_\text{W} + C_\text{Kal}) = T_\text{GG}(C_\text{W} + C_\text{Kal} + C_\text{Al}) \\
  \Rightarrow C_\text{Al} (T_\text{Al} - T_\text{GG}) & = (C_\text{W} + C_\text{Kal})(T_\text{GG} - T_\text{W})\\
  \Rightarrow C_\text{Al}& = \frac{(C_\text{W} + C_\text{Kal})(T_\text{GG} - T_\text{W})}{T_\text{Al} - T_\text{GG}} \\
  c_\text{Al} & = \frac{(m_\text{W}c_\text{W}+ C_\text{Kal})(T_\text{GG} - T_\text{W})}{(T_\text{Al} - T_\text{GG}) lbh \varrho} \\
  &= \frac{(\unit[0.2]{kg} \cdot \unit[4190]{\frac{J}{kg \cdot ^\circ C} + \unit[209]{\frac{J}{^\circ C})(\unit[24.1]{^\circ C} - \unit[17]{^\circ C})}}}{(\unit[100]{^\circ C} - \unit[24.1]{^\circ C})(\unit[0.05]{m} \cdot \unit[0.04]{m} \cdot\unit[0.02]{m} \cdot \unit[2720]{\frac{kg}{m^3}})} \\
  &= \unit[900.2]{\frac{J}{kg\cdot ^\circ C}}
\end{align*}

Wikipedia sagt $c_\text{Al} = \unit[897]{\frac{J}{kg\cdot K}}$.

\subsection{zu \nameref{aufg:Waermeleitung}}
\label{lsg:Waermeleitung}

Der Wärmestrom durch einen Körper der Länge $l$ und des Querschitts $A$ ist
\begin{align*}
  \Phi = \frac{\lambda A}{l}\Delta T =: \frac{\Delta T}{R_\lambda}
\end{align*} mit dem Wärmeleitwiderstand $R_\lambda$. Wie elektrische Widerstände addieren sich Wärmeleitwiderstände bei "`Reihenschaltung"', so dass sich
\begin{align*}
  R_\text{ges} = \frac{1}{A} \left( \frac{l_1}{\lambda_1} + \frac{l_2}{\lambda_2}\right)
\end{align*} ergibt. Für das Verhältnis der Wärmeströme mit und ohne Isolierung ergibt sich also
\begin{align*}
  \frac{\Phi_\text{iso}}{\Phi_\text{ohne}} & = \frac{\Delta T A \cdot l_1}{\left( \frac{l_1}{\lambda_1} + \frac{l_2}{\lambda_2} \right) \cdot A\Delta T  \lambda_1} = \frac{l_1 \lambda_1\lambda_2}{(l_1\lambda_2 + l_2\lambda_1)\lambda_1} = \frac{l_1\lambda_2}{l_1\lambda_2 + l_2\lambda_1} \\
  & = \frac{\unit[0.2]{m} \cdot \unit[0.3]{\frac{W}{m\cdot K}}}{\unit[0.2]{m} \cdot \unit[0.3]{\frac{W}{m\cdot K}} + \unit[0.02]{m} \cdot \unit[2.1]{\frac{W}{m\cdot K}}} = 0.5882
\end{align*}

\subsection{zu \nameref{aufg:Doppler}}
\label{lsg:Doppler}

Für den relativistischen Dopplereffekt gilt mit $v>0$ bei Annäherung
\begin{align*}
  f_\text{B} = f_\text{Q} \sqrt{\frac{c+v}{c-v}}
\end{align*}
Die Schwebungsfrequenz ergibt sich aus der Differenz der beteiligten Frequenzen:
\begin{align*}
  f_\text{Schweb} = f_\text{dir} - f_\text{indir} & = f_0\sqrt{\frac{c+v}{c-v}} - f_0\sqrt{\frac{c-v}{c+v}}\\
  &= \frac{c}{\lambda} \left( \sqrt{\frac{c+v}{c-v}} - \sqrt{\frac{c-v}{c+v}} \right)\\
  &= \frac{c}{\unit[632.8\cdot 10^{-9}]{m}} \cdot 2.4\cdot 10^{-7}\\
  &= \unit[1.138 \cdot 10^8]{Hz}
\end{align*}

Die gesuchte Geschwindigkeit ergibt sich durch Umstellung nach $v$ mit der Definition $a := \frac{f_\text{Schweb}}{f_0}$:
\begin{align*}
  a & = \sqrt{\frac{c+v}{c-v}} - \sqrt{\frac{c-v}{c+v}} \\
  \Rightarrow a^2 & = \frac{c+v}{c-v} - 2\sqrt{\frac{(c+v)(c-v)}{(c-v)(c+v)}} + \frac{c-v}{c+v} \\
  \Rightarrow a^2 + 2 & = \frac{(c+ v)^2 + (c-v)2}{(c-v)(c+v)}\\
  \Rightarrow a^2 + 2 & = \frac{c^2 + 2cv + v^2 + c^2 - 2cv + v^2}{c^2 - v^2}\\
  \Rightarrow (a^2 + 2)(c^2-v^2) &= 2c^2 + 2v^2\\
  \Rightarrow a^2c^2 + 2c^2 - a^2v^2 - 2v^2 & = 2c^2 + 2v^2\\
  \Rightarrow a^2c^2 & = (a^2+4)v^2\\
  \Rightarrow v = \frac{ac}{\sqrt{a^2+4}}
\end{align*}

\end{document}
